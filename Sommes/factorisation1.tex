\begin{lemma}
\label{somme:factorisation}
Soient $a, b \in \setComplexNumbers$ et $\forall n \in \setNaturalNumbers^*$, alors 
\[
a^n - b^n = (a-b)\times \displaystyle\sum\limits_{k=0}^{n-1} a^k b^{n - 1 - k}
\]
\end{lemma}

\begin{proof} 
Cette égalité peut se prouver de deux manière différentes

\textbf{Méthode 1 : En développant les différents termes}\\
Soient $a, b \in \setComplexNumbers$, soit $n \in \setNaturalNumbers^*$, alors par linéarité des sommes, on a 
\[
(a-b)\times \displaystyle\sum\limits_{k=0}^{n-1} a^k b^{n - 1 - k}
=a \times \left(\displaystyle\sum\limits_{k=0}^{n-1} a^{k} b^{n - 1 - k}\right) - b \times \left(\displaystyle\sum\limits_{k=0}^{n-1} a^k b^{n - 1 - k}\right)
\]
Donc par distributivité, on peut rentrer dans les sommes les facteurs $a$ et $b$ :
\[
(a-b)\times \displaystyle\sum\limits_{k=0}^{n-1} a^k b^{n - 1 - k}
=\left(\displaystyle\sum\limits_{k=0}^{n-1} a^{k+1} b^{n - 1 - k}\right) - \left(\displaystyle\sum\limits_{k=0}^{n-1} a^k b^{n - k}\right)
\]
On effectue le changement d'indice dans la première somme $k \leftarrow k+1$, ce qui donne
\[
(a-b)\times \displaystyle\sum\limits_{k=0}^{n-1} a^k b^{n - 1 - k}
=
\left(\displaystyle\sum\limits_{k=1}^{n} a^{k} b^{n - k}\right) - \left(\displaystyle\sum\limits_{k=0}^{n-1} a^k b^{n - k}\right)
\]
Les termes des deux sommes se télescopent, on obtient alors 
\[
\lboxed{(a-b)\times \displaystyle\sum\limits_{k=0}^{n-1} a^k b^{n - 1 - k}}
=
a^{n} \underbrace{b^{n - n}}_{=1} - \underbrace{a^{0}}_{=1} b^{n - 0}
\rboxed{= a^n - b^n}
\]

\textbf{Méthode 2 : Par récurrence}\\
Soient $a, b \in \setComplexNumbers$, on définit 
\[
\forall n \in \setNaturalNumbers^*,
\mathcal{P}(n) : 
(a-b)\times \displaystyle\sum\limits_{k=0}^{n-1} a^k b^{n - 1 - k}  
=  a^{n}  - b^{n}
\]

\textbf{Initialisation :} Pour $n = 1$, on a 
\[
(a-b)\times \displaystyle\sum\limits_{k=0}^{0} a^k b^{0 - k} 
=(a - b) \times  \underbrace{a^0 b^{0}}_{=1} 
= a - b
\]
Et d'autre part
\[
a^{1} - b^{1} = a - b
\]
Donc $\mathcal{P}(1)$ est donc vraie. 

\textbf{Hérédité :} On suppose que l'assertion est vraie pour un certain rang $n \in \setNaturalNumbers^*$ et on montre 
\[
\mathcal{P}(n+1) :
(a-b)\times \displaystyle\sum\limits_{k=0}^{n} a^k b^{n - k} 
=  a^{n+1}  - b^{n+1}
\]
On a 
\[
a^{n+1}  - b^{n+1} 
= a \times a^{n}  - b^{n+1}
\]
On va maintenant faire apparaître l'hypothèse de récurrence en rajoutant de nouveaux termes :
\[
a^{n+1}  - b^{n+1} 
=a \times a^{n} \underbrace{+ a \times b^{n} - a \times b^{n}}_{=0} +  - b^{n+1}
=a \times (a^{n} - b^{n}) + a \times b^{n} - b^{n+1}
\]
Par hypothèse de récurrence, on obtient 
\[
= a \times (a-b)\times \left(\displaystyle\sum\limits_{k=0}^{n-1} a^k b^{n - 1 - k}\right) + a \times b^{n} - b^{n+1}
\]
Donc, en rentrant le facteur $a$ dans la somme et en effectuant un changement d'indice $k \leftarrow k+1$, on a
\[
a^{n+1}  - b^{n+1} = (a-b)\times \left(\displaystyle\sum\limits_{k=1}^{n} a^{k} b^{n  - k}\right) +  a \times b^{n} - b^{n+1}
\]
On factorise le second terme par $(a-b)$
\[
a^{n+1}  - b^{n+1}
= (a-b)\times \left(\displaystyle\sum\limits_{k=1}^{n} a^{k} b^{n  - k}\right) + (a - b) \times b^{n}
\]
On en déduit que 
\[
a^{n+1}  - b^{n+1}
= (a-b)\times \left(\underbrace{b^{n}}_{=a^0 b^n} + \displaystyle\sum\limits_{k=1}^{n} a^{k} b^{n  - k}\right)
\]

On remarque que le terme d'indice $k=0$ est apparu car $a^0=1$, on va donc le rentrer dans la somme. Cela nous donne ainsi
\[
a^{n+1}  - b^{n+1}
= (a-b)\times \displaystyle\sum\limits_{k=0}^{n} a^k b^{n - k}
\]
Nous avons donc montré que si $\mathcal{P}(n)$ est vraie alors $\mathcal{P}(n+1)$ est vraie aussi.

\textbf{Conclusion :}  
\[
\lrboxed{\forall n \in \setNaturalNumbers^*,
\mathcal{P}(n) : 
(a-b)\times \displaystyle\sum\limits_{k=0}^{n-1} a^k b^{n - 1 - k}  
=  a^{n}  - b^{n}}
\]
\end{proof}
