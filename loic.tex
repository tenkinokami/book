\begin{exercice}
Soit $f : 
\left\{
\begin{array}{l l l}
x &\text{ si } &x \in [ 1, e [\\
x f(\ln(x)) &\text{ si }& x > e\\
\end{array} 
\right.$

La série $\displaystyle\sum\limits_{n = 1}^{+\infty} \dfrac{1}{f(n)}$ est-elle convergente ?
\end{exercice}

\begin{proof}
On remarque que $f$ est une fonction définie de manière récursive. \\
Sur l'intervalle $[ 1, e [ = [\exp^{0}(1), \exp^{1}(1)[$, $f$ est la fonction identité $x \longmapsto x$\\
Sur l'intervalle $[ e, e^e [ = [\exp^{1}(1), \exp^{2}(1)[$, $f$ est la fonction $x \longmapsto f(\ln(x)) = x \ln(x)$ car $\ln(x) \in [1, e[$\\
Sur l'intervalle $[ e^e, e^{e^e} [ = [\exp^{2}(1), \exp^{3}(1)[$, $f$ est la fonction $x \longmapsto x \ln(x) \ln(\ln(x))$ en appliquant le logarithme deux fois récursivement.\\
Ainsi, pour tout $n \in \mathbb{N}$, si $x \in [\exp^{n}(1), \exp^{n+1}(1)[$, alors $f(x)= \displaystyle\prod\limits_{i = 0}^{n} \ln^{i}(x)$.\\

On sait par récurrence que la primitive de $\dfrac{1}{f}$ sur chaque segment est $\ln^{n+1}(x)$, donc on a :
\[
\displaystyle\int\limits_{1}^{+\infty} \dfrac{1}{f(x)} \: d x 
= \displaystyle\sum\limits_{n = 0}^{+\infty} \displaystyle\int\limits_{\exp^{n}(1)}^{\exp^{n+1}(1)} \dfrac{1}{f(x)} \: d x 
= \displaystyle\sum\limits_{n = 0}^{+\infty} \displaystyle\int\limits_{\exp^{n}(1)}^{\exp^{n+1}(1)} \dfrac{1}{\displaystyle\prod\limits_{i = 0}^{n} \ln^{i}(x)} \: d x 
\]
\[
=\displaystyle\sum\limits_{n = 0}^{+\infty} \Bigl[\ln^{n+1}(x)\Bigr]_{\exp^{n}(1)}^{\exp^{n+1}(1)} 
= \displaystyle\sum\limits_{n = 0}^{+\infty} \Bigr(\ln^{n+1}(\exp^{n+1}(1)) - \ln^{n+1}(\exp^{n}(1))\Bigr)
\]
\[
= \displaystyle\sum\limits_{n = 0}^{+\infty} \Bigr(1 - \ln(\exp^{0}(1))\Bigr) 
= \displaystyle\sum\limits_{n = 0}^{+\infty} 1
\]

Cette dernière série étant divergente grossièrement, on en déduit que $\displaystyle\int\limits_{1}^{+\infty} \dfrac{1}{f(x)} \: d x $ est divergente.
Cependant, $\dfrac{1}{f}$ étant une fonction monotone, donc $\displaystyle\int\limits_{1}^{+\infty} \dfrac{1}{f(x)} \: d x $ et  $\displaystyle\sum\limits_{n = 1}^{+\infty} \dfrac{1}{f(n)}$ ont la même monotonie. Ainsi, la série $\displaystyle\sum\limits_{n = 1}^{+\infty} \dfrac{1}{f(n)}$ est une série divergente.  

\textbf{Méthode alternative}\\
Plus astucieusement, on peut remarquer que :
\[
\displaystyle\int\limits_{1}^{+\infty} \dfrac{1}{f(x)} \: d x 
= \displaystyle\int\limits_{1}^{e} \dfrac{1}{f(x)} \: d x + \displaystyle\int\limits_{e}^{+\infty} \dfrac{1}{f(x)} \: d x
= \displaystyle\int\limits_{1}^{e} \dfrac{1}{f(x)} \: d x + \displaystyle\int\limits_{e}^{+\infty} \dfrac{1}{x f(\ln(x))} \: d x
\]
On peut alors effectuer le changement de variable $t = \ln(x)$ dans la seconde intégrale, ce qui nous donne :
\[
\displaystyle\int\limits_{e}^{+\infty} \dfrac{1}{x f(\ln(x))} \: d x = \displaystyle\int\limits_{1}^{+\infty} \dfrac{1}{e^t f(t)} \: e^t d t = \displaystyle\int\limits_{1}^{+\infty} \dfrac{1}{f(t)} d t
\]
Cela signifie alors que la première intégrale $\displaystyle\int\limits_{1}^{e} \dfrac{1}{f(x)} \: d x$ est nulle, ce qui est absurde si on suppose qu'elle était convergente. La fonction intégrée étant positive, on en déduit que $\displaystyle\int\limits_{1}^{+\infty} \dfrac{1}{f(x)} \: d x$ est infinie, et on peut ensuite conclure comme dans la première méthode
\end{proof}