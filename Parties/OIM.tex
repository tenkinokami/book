\begin{exercice}
Trouver toutes les fonctions $f \in \setFunctions{\setRealNumbers}{\setRealNumbers}$ qui vérifient la propriété fonctionnelle $P$
\[
\forall x, y \in \setRealNumbers, P(x, y): f(f(x)f(y))+f(x+y)=f(xy)
\]
\end{exercice}

\begin{proof}
On raisonne par analyse synthèse.

\textbf{Analyse :} Supposons que $f$ vérifie la propriété fonctionnelle $P$, alors :\\
\[
P(0, 0) : f(f(0)f(0)) + f(0 + 0) = f(0 \times 0)
\]
Donc en posant $a = f(0)$, on obtient :
\[
f(a^2) + a = a \quad\text{c'est à dire}\quad f(a^2)=0
\]
Ainsi, $a^2$ est un des zéros de la fonction. La propriété fonctionnelle nous donne aussi que :
\[
\forall x \in \setRealNumbers, P(x, a^2) : f(f(x)f(a^2)) + f(x + a^2)= f(x a^2)
\]
Cela revient a dire que :
\[
\forall x \in \setRealNumbers, f(a^2 x) - f(x + a^2) = f( x \times 0) = f(0) = a
\]
On va alors distinguer plusieurs cas : 
\begin{itemize}
\item Si $a = 0$, alors on a $\forall x \in \setRealNumbers, f(0 x) - f(x + 0) = 0$, autrement dit, $f$ est la fonction nulle $\function{x}{0}$.
\item Si $a = 1$ ou $-1$, alors on a $\forall x \in \setRealNumbers, f(1 x) - f(x + 1) = a$, donc $f(x + 1) = f(x) - a$. On retombe sur une équation fonctionnelle classique, les solutions de cette dernières sont des fonctions affines de la forme $\function{x}{A x + B}$ où $A$ et $B$ est des constantes à déterminer.

Ici, on a $A = -a$ et B tel que $a = f(0) = \pm 1$, donc $B$ vérifie :
\[
f(0) = -a \times 0 + B = a \quad\text{ ce qui revient à dire que }\quad B = a \quad\text{convient.}
\]
Ainsi, seuls les fonctions $\function{x}{-x + 1}$ et $\function{x}{x - 1}$ peuvent convenir dans ce cas. 

\item Enfin, le dernier comprends tous les autres ($a \in \setRealNumbers \setminus \set{-1, 0, 1}$). \\
Cherchons une valeur de $x$ pour laquelle $a^2 x = x + a^2$. Cette égalité donne $(a^2 - 1)x = a^2$ donc $x = \dfrac{a^2}{a^2 - 1}$. \\
On note alors $A = f\left(\dfrac{a^2}{a^2 - 1} + a^2\right) = f\left(\dfrac{a^2}{a^2 - 1} \times a^2\right)$.En réinjectant directement dans la dernière équation obtenue on obtient :
\[
\forall x \in \setRealNumbers, A - A = a \quad\text{ donc }\quad a = 0
\]
Ce qui est absurde comme on se plaçait dans le cas $a \not\in \set{-1, 0, 1}$
\end{itemize}
Ainsi, si $f$ est vérifie $P$, alors $f\in \set{x \longmapsto 0, x \longmapsto - x + 1, x \longmapsto x - 1}$.

\textbf{Synthèse :} Soit $\function[f]{x}{0}$, alors :
\[
\forall x, y \in \setRealNumbers, f(f(x)f(y))+f(x+y) = 0 + 0 = 0 \quad\text{ et }\quad f(xy) = 0
\]
Donc $P$ est vérifiée et $x \longmapsto 0$ est bien solution du problème.\\
Soit $f = x \longmapsto - x + 1$, alors, on a 
\[
\begin{aligned}
\forall x, y \in \setRealNumbers, 
f(f(x)f(y))+f(x+y) 
&= (-((- x + 1)(- y + 1)) + 1) + (-(x+y) + 1)\\
&= -(xy - x - y + 1) + 1 -(x+y) + 1\\
&= -xy  + 1
\end{aligned}
\]
D'autre part, on a $\forall x, y \in \setRealNumbers, f(xy) = - xy + 1$, donc $P$ est vérifiée et $x \longmapsto - x + 1$ est bien solution du problème.\\ 
Enfin, si $f = x - 1$, alors
\[
\begin{aligned}
\forall x, y \in \setRealNumbers, 
f(f(x)f(y))+f(x+y) 
&= (((x - 1)(y - 1)) - 1) + ((x+y) - 1)\\
&= (xy - x - y + 1) - 1 +(x+y) - 1\\
&= xy  - 1 \\
&= f(xy)
\end{aligned}
\]
$P$ est vérifiée et $x \longmapsto x - 1$ est bien solution du problème.

\textbf{Conclusion :} Ainsi, l'ensemble des solutions vérifiant $P$ est \[
\set{x \longmapsto 0, x \longmapsto - x + 1, x \longmapsto x - 1}
\]
\end{proof}

\begin{exercice}
Pour tout entier $a_0 > 1$, on définit la suite $(a_n)_{n \in \setNaturalNumbers}$ par :
\[
\forall n \in \setNaturalNumbers, a_{n+1} = 
\left\{
\begin{array}{l l}
\sqrt{a_n} &\text{ si }  \sqrt{a_{n}} \in \setNaturalNumbers\\
a_{n} + 3  &\text{ sinon}\\
\end{array} 
\right.
\]
Déterminer toutes les valeurs de $a_0$ pour lesquelles il existe un nombre $A$ tel que $a_n = A$ pour une infinité de valeurs de $n$.
\end{exercice}

\begin{proof}
Pour déterminer les valeurs pour lesquelles il existe un nombre $A$ tel que $a_n = A$ pour une infinité de valeurs de $n$, on va distinguer $3$ cas : $a_0 \equiv 0 [3]$, $a_0 \equiv 1 [3]$ et $a_0 \equiv 2 [3]$. En réalité, le fait qu'il s'agisse du terme d'indice $0$ n'est pas important dans ce problème car la suite est récurrente d'ordre $1$. On va donc noter $N$ la valeur que prend la suite à l'étape $N$.

\textbf{Premier cas :} Si $a_{N} \equiv 2 [3]$, alors, par théorème, $a_N$ n'est pas un carré parfait, donc le terme suivant est $a_{N+1} = a_{N} + 3 \equiv 2 [3]$. Par récurrence, on obtient que $\forall n \in \setNaturalNumbers, a_{N+n} = a_{N} + 3 (n - N)$. La suite diverge vers l'infini et ne se répète plus à partir du rang $N$. 

\textbf{Deuxième cas :} Si $a_{N} \equiv 0 [3]$
\begin{itemize}
\item Si $a_{N}$ est un carré parfait alors $a_{N+1} \equiv 0 [3]$.
\item Si $a_{N}$ n'est pas un carré parfait, alors $a_{N+1} = a_{N} + 3$. 
\end{itemize}
Ainsi, si $a_{N}$ est dans ce cas, les termes suivants aussi. Par théorème, si $a_{N} > 3$ il existe $T$ tel que $a_{N+T}$ soit un carré parfait strictement plus petit que $a_{N}^2$ et multiple de $3$ donc $a_{N+T+1} < a_{N}$. Il existe donc une suite d'indice pour lesquelles $f$ est décroissante. Donc il existe un rang à partir duquel la suite vaut $3$.

Si $a_{N} = 3$ alors $a_{n+1} = 6$ et $a_{N+2} = 9$ et $a_{N+3} = 3$, la suite est donc périodique et prend une infinité de fois les valeurs $3$, $6$ et $9$.

\textbf{Troisième cas :} Si $a_{N} \equiv 1 [3]$, alors le carré immédiatement au dessus vaut soit $1$ soit $2$ modulo $3$. 
\begin{itemize}
\item S'il vaut $2$ modulo $3$, on retombe dans le premier cas et on diverge vers l'infini
\item Si a chaque fois, le carré qu'on atteint vaut $1$ modulo $3$, alors la suite est décroissante et le dernier atteint est $4$. Or $4$ donne $7, 10, 13, 16, 4$ donc les valeurs $4$, $7$, $10$, $13$ et $16$ se répètent une infinité de fois.  
\end{itemize}
\end{proof}