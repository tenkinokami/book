\begin{lemma}
Soit $n \in \setNaturalNumbers$, alors 
\begin{itemize}
\item $3 n + 2$ n'est pas un carré parfait
\item Si $3 n$ est un carré parfait, alors $\sqrt{3 n}$ est divisible par $3$. 
\end{itemize}
\end{lemma}

\begin{proof}
Soit $i \in \setNaturalNumbers$, alors : 
\begin{itemize}
\item $i \equiv 0 [3]$, donc $i^2 = 0[3]$
\item $i \equiv 1 [3]$, donc $i^2 = 1[3]$
\item $i \equiv 2 [3]$, donc $i^2 = 4 = 1[3]$
\end{itemize}
Ainsi, il n'existe pas de valeur de $i$ telle que $i^2 = 2[3]$ donc tous les nombres de la forme $(3n +2)_{n \in \setNaturalNumbers}$ ne peuvent pas être des carrés parfaits. \\
D'autre part, le seul antécédent modulo $3$ de $0$ est $0$ modulo $3$, on en déduit la seconde propriété.
\end{proof}