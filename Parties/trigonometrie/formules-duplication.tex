\begin{lemma}[Duplication du cosinus]
\label{trigonometrie:duplication:cosinus_sinus_tangente}
Soit $a \in \setRealNumbers$, alors 
\[
\cos(2a) = \cos^2(a) - \sin^2(a)=2 \cos^2(a) - 1= 1 - 2 \sin^2(a)
\]
\[
\sin(2a) = 2\sin(a)\cos(a)
\]
\[
\lboxed{\tan(2a)}=\tan(a+a) = \dfrac{\tan(a)+\tan(a)}{1-\tan(a)\tan(a)} \rboxed{= \dfrac{2\tan(a)}{1-\tan^2(a)}}
\]
\end{lemma}

\begin{proof}
Soit $a \in \setRealNumbers$, alors on peut appliquer les formules d'addition du théorème \ref{trigonometrie:addition}. Pour le cosinus, cela donne :
\[
\lboxed{\cos(2a)} = \cos(a + a)=\cos(a)\cos(a) - \sin(a)\sin(a) \rboxed{= \cos^2(a) - \sin^2(a)}
\]
On obtient les deux autres formules en appliquant l'identité $ \cos^2(a) + \sin^2(a) = 1$ et en replaçant les termes
\[
\lboxed{\cos(2a)} =  \cos^2(a) - \sin^2(a) =  \cos^2(a) - (1 - \cos^2(a)) \rboxed{= 2 \cos^2(a) - 1}
\]
Et d'autre part
\[
\lboxed{\cos(2a)} =  \cos^2(a) - \sin^2(a) =  (1 - \sin^2(a)) - \sin^2(a) \rboxed{= 1  - 2 \sin^2(a)}
\]

Soit $a \in \setRealNumbers$, de la même manière que pour le cosinus, on a d'après le théorème \ref{trigonometrie:addition} :
\[
\lboxed{\sin(2a)}=\sin(a+a) = \cos(a)\sin(a) + \sin(a)\cos(a) \rboxed{= 2\cos(a)\sin(a)}
\]

Soit $a \in \setRealNumbers$, de la même manière que précédemment, on a d'après le théorème \ref{trigonometrie:addition} :
\[
\lboxed{\tan(2a)}=\tan(a+a) = \dfrac{\tan(a)+\tan(a)}{1-\tan(a)\tan(a)} \rboxed{= \dfrac{2\tan(a)}{1-\tan^2(a)}}
\]
\end{proof}