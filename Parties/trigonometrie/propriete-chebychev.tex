\begin{property}
Pour tout $n \in \setNaturalNumbers$ et pour tout $x \in \setRealNumbers$, on a
\[
\polChebCos{n}(\cos(x))=\cos(nx) 
\quad\quad
\polChebSin{n}(\sin(x))=\dfrac{\sin((n+1)x)}{\sin(x)}
\]
\end{property}

\begin{proof}
Soit $x \in \setRealNumbers$. On raisonne par \textbf{récurrence double}. On définit pour tout $n \in \setNaturalNumbers$ la propriété
\[
\mathcal{P}(n) : \:\polChebCos{n}(\cos(x))=\cos(nx) 
\]
\textbf{Initialisation :} Pour $n=0$, on a $\polChebCos{0}(\cos(x)) = 1$ et $\cos(0x) =\cos(0) = 1$ donc $\mathcal{P}(0)$ est vérifiée.\\
De la même manière, $\polChebCos{1}(\cos(x)) = \cos(x)$ et $\cos(1x)=\cos(x)$ donc $\mathcal{P}(1)$ est vérifiée. 

\textbf{Hérédité :} On suppose que la propriété est vraie pour un certain rang $n$ et $n+1$ et on démontre que $\mathcal{P}(n)$ est vraie :

Comme $\polChebCos{n+2}$ est un polynôme de Tchebychev de première espèce, on a 
\[
\polChebCos{n+2}(X) = 2 X \polChebCos{n+1}(X) - \polChebCos{n}(X)
\]
Par hypothèse de récurrence, $\polChebCos{n+1}(\cos(x)) = \cos((n+1)x)$ et $\polChebCos{n}(\cos(x)) = \cos(nx)$, donc en réinjectant dans la formule de récurrence, on a 
\[
\polChebCos{n+2}(\cos(x)) = 2 \cos(x)\cos((n+1)x) - \cos(nx)
\]
Or d'après les formules de linéarisation, on a 
\[
\cos(x)\cos((n+1)x) = \dfrac{1}{2}\Bigl(\cos(nx)+\cos((n+2)x)\Bigr)
\]
On a donc 
\[
\lboxed{\polChebCos{n+2}(\cos(x))} = \Bigl(\cos(nx)+\cos((n+2)x)\Bigr) - \cos(nx) \rboxed{= \cos((n+2)x)}
\]
\textbf{Conclusion :} On a donc \fbox{$\forall n \in \setNaturalNumbers, \:\polChebCos{n}(\cos(x))=\cos(nx)$}. 
\end{proof}

\begin{proof}
Soit $x \in \setRealNumbers$. On raisonne par \textbf{récurrence double}. On définit pour tout $n \in \setNaturalNumbers$ la propriété
\[
\mathcal{P}(n) : \:\polChebSin{n}(\cos(x))=\dfrac{\sin((n+1)x)}{\sin(x)}
\]
\textbf{Initialisation :} Pour $n=0$, on a $\polChebSin{0}(\cos(x)) = 1$ et $\dfrac{\sin((0+1)x)}{\sin(x)} =\dfrac{\sin(x)}{\sin(x)}  = 1$ donc $\mathcal{P}(0)$ est vérifiée.\\
De la même manière, $\polChebSin{1}(\cos(x)) = 2\cos(x)$ et $\dfrac{\sin((1+1)x)}{\sin(x)}=\dfrac{\sin(2x)}{\sin(x)}$. Or on sait d'après les formules de duplication que 
\[
\sin(2x)=2\sin(x)\cos(x)
\]
Donc $\mathcal{P}(1)$ est vérifiée. 

\textbf{Hérédité :} On suppose que la propriété est vraie pour un certain rang $n$ et $n+1$ et on démontre que $\mathcal{P}(n)$ est vraie :

Comme $\polChebSin{n+2}$ est un polynôme de Tchebychev de deuxième espèce, on a 
\[
\polChebSin{n+2}(X) = 2 X \polChebSin{n+1}(X) - \polChebSin{n}(X)
\]
Par hypothèse de récurrence, $\polChebSin{n+1}(\cos(x)) = \dfrac{\sin((n+2)x)}{\sin(x)}$ et $\polChebSin{n}(\cos(x)) = \dfrac{\sin((n+1)x)}{\sin(x)}$, donc en réinjectant dans la formule de récurrence, on a
\[
\sin(x)\polChebCos{n+2}(\cos(x)) = 2 \cos(x)\sin((n+2)x) - \sin((n+1)x)
\]
Or d'après les formules de linéarisation, on a 
\[
\sin((n+2)x)\cos(x) = \dfrac{1}{2}\Bigl(\sin((n+3)x)+\sin((n+1)x)\Bigr)
\]
On a donc 
\[
\sin(x)\polChebSin{n+2}(\cos(x)) = \Bigl(\sin((n+3)x)+\sin((n+1)x)\Bigr) - \sin((n+1)x) = \sin((n+3)x)
\]
Ce qui finalement revient à dire que
\[
\lrboxed{\polChebSin{n+2}(\cos(x)) = \dfrac{\sin((n+3)x)}{\sin(x)}}
\]
\textbf{Conclusion :} On a donc $\lrboxed{\forall n \in \setNaturalNumbers, \:\polChebSin{n}(\cos(x)) = \dfrac{\sin((n+1)x)}{\sin(x)}}$. 
\end{proof}
