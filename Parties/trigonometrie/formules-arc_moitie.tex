\begin{lemma}
\label{trigonometrie:arc_moitie}
Soit $x \in \setRealNumbers$, on note $t = \tan\left(\dfrac{x}{2}\right)$, alors 
\[
\cos(x) = \dfrac{1-t^2}{1+t^2}
\qquad\quad
\sin(x)= \dfrac{2t}{1+t^2}
\qquad\quad
\tan(x) = \dfrac{2t}{1-t^2}
\]
\end{lemma}

\begin{proof}
Soit $x \in \setRealNumbers$, alors en appliquant les formules de duplication :
\[
\cos(x) 
= \cos^2\left(\dfrac{x}{2}\right)-\sin^2\left(\dfrac{x}{2}\right) 
= \dfrac{\cos^2\left(\dfrac{x}{2}\right)-\sin^2\left(\dfrac{x}{2}\right)}{1}
\]
Or on sait que $\cos^2\left(\dfrac{x}{2}\right)+\sin^2\left(\dfrac{x}{2}\right) = 1$, donc
\[
\cos(x)=\dfrac{\cos^2\left(\dfrac{x}{2}\right)-\sin^2\left(\dfrac{x}{2}\right)}{\cos^2\left(\dfrac{x}{2}\right)+\sin^2\left(\dfrac{x}{2}\right)}
\]
En divisant par $\cos^2\left(\dfrac{x}{2}\right)$ au numérateur et au dénominateur, on a 
\[
\lboxed{\cos(x)}
=\dfrac{\dfrac{\cos^2\left(\dfrac{x}{2}\right)}{\cos^2\left(\dfrac{x}{2}\right)}-\dfrac{\sin^2\left(\dfrac{x}{2}\right)}{\cos^2\left(\dfrac{x}{2}\right)}}{\dfrac{\cos^2\left(\dfrac{x}{2}\right)}{\cos^2\left(\dfrac{x}{2}\right)}+\dfrac{\sin^2\left(\dfrac{x}{2}\right)}{\cos^2\left(\dfrac{x}{2}\right)}}
=\dfrac{1-\tan^2\left(\dfrac{x}{2}\right)}{1+\tan^2\left(\dfrac{x}{2}\right)}
\rboxed{=\dfrac{1-t^2}{1+t^2}}
\]

De la même manière,
\[
\sin(x) = 2\cos\left(\dfrac{x}{2}\right)\sin\left(\dfrac{x}{2}\right) =\dfrac{2\cos\left(\dfrac{x}{2}\right)\sin\left(\dfrac{x}{2}\right)}{1}
\]
Donc 
\[
\sin(x) =\dfrac{2\cos\left(\dfrac{x}{2}\right)\sin\left(\dfrac{x}{2}\right)}{\cos^2\left(\dfrac{x}{2}\right)+\sin^2\left(\dfrac{x}{2}\right)}
\]
En divisant par $\cos^2\left(\dfrac{x}{2}\right)$, on a 
\[
\lboxed{\sin(x)} =
\dfrac{\dfrac{2\cos\left(\dfrac{x}{2}\right)\sin\left(\dfrac{x}{2}\right)}{\cos^2\left(\dfrac{x}{2}\right)}}{\dfrac{\cos^2\left(\dfrac{x}{2}\right)}{\cos^2\left(\dfrac{x}{2}\right)}+\dfrac{\sin^2\left(\dfrac{x}{2}\right)}{\cos^2\left(\dfrac{x}{2}\right)}}
=\dfrac{\dfrac{2\sin\left(\dfrac{x}{2}\right)}{\cos\left(\dfrac{x}{2}\right)}}{1+\tan^2\left(\dfrac{x}{2}\right)}
= \dfrac{2\tan\left(\dfrac{x}{2}\right)}{1+\tan^2\left(\dfrac{x}{2}\right)}
\:\rboxed{= \dfrac{2t}{1+t^2}}
\]

Enfin, par définition de la tangente
\[
\lboxed{\tan(x)}
=\dfrac{\sin(x)}{\cos(x)} 
= \dfrac{\dfrac{2t}{1+t^2}}{\dfrac{1-t^2}{1+t^2}} 
\:\rboxed{= \dfrac{2t}{1-t^2}}
\]
Ce résultat peut aussi se retrouver directement par la formule de duplication de la tangente
\[
\tan(x)
=\tan\left(2 \times \dfrac{x}{2}\right)
=\dfrac{2\tan\left(2 \times \dfrac{x}{2}\right)}{1 - \tan^2\left(2 \times \dfrac{x}{2}\right)} 
= \dfrac{2t}{1-t^2}
\]
\end{proof}