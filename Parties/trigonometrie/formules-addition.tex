\begin{theorem}[Formules d'addition des cosinus et sinus]
\label{trigonometrie:addition:sinus_et_cosinus}
Soient $a, b \in \setRealNumbers$, alors
\[
\cos(a+b) = \cos(a)\cos(b) - \sin(a)\sin(b) 
\qquad\qquad
\sin(a+b) = \sin(a)\cos(b) + \cos(a)\sin(b)
\]
\[
\cos(a-b) = \cos(a)\cos(b) + \sin(a)\sin(b)
\qquad\qquad
\sin(a-b) = \sin(a)\cos(b) - \cos(a)\sin(b)
\]
\end{theorem}

\begin{proof}
Soient $a, b \in \setRealNumbers$, alors d'après les propriétés de la fonction exponentielle, on sait que 
\[
e^{i\:(a+b)} =e^{i\:a}\times e^{i\:b}
\]
En utilisant la forme polaire des nombres complexes, on a
\[
e^{i\:a}\times e^{i\:b} 
= \left(\cos(a)+i\:\sin(a)\right)\left(\cos(b)+i\:\sin(b)\right)
\]
En développant le double produit, on obtient
\[
e^{i\:a}\times e^{i\:b}
=  \cos(a)\cos(b) +i\:\cos(a)\sin(b)+i\:\sin(a)\cos(b)+i^2\:\sin(a)\sin(b)
\]
En utilisant le fait que $i^2 = 1$, on peut alors séparer les termes réels et les termes imaginaires
\[
e^{i\:a}\times e^{i\:b} = \left(\cos(a)\cos(b)-\sin{a}\sin{b}\right)+i\:\left(\sin(a)\cos(b) +\cos(a)\sin(b)\right)
\]
Donc
\[\realPart{e^{i\:a}\times e^{i\:b}} = \cos(a+b) = \cos(a)\cos(b) - \sin(a)\sin(b)
\]
Et
\[
\imaginaryPart{e^{i\:a}\times e^{i\:b}} = \sin(a+b) = \sin(a)\cos(b) + \cos(a)\sin(b)
\]
Comme $e^{i\:(a+b)} = e^{i\:a}\times e^{i\:b}$, on peut identifier les parties réelles et les parties complexes, ce qui donne bien
\[
\lrboxed{\cos(a+b) = \cos(a)\cos(b) - \sin(a)\sin(b)}
\]
\[
\lrboxed{\sin(a+b) = \sin(a)\cos(b) + \cos(a)\sin(b)}
\]
En transformant $b$ en $-b$ dans l'équation , on a alors 
\[
\cos(a-b) = \cos(a)\cos(-b) - \sin(a)\sin(-b)
\]
\[
\sin(a-b) = \sin(a)\cos(-b) + \cos(a)\sin(-b)
\]
La fonction sinus est impaire donc $\sin(b)= -\sin(b)$ et la fonction cosinus est paire, donc $\cos(b)=\cos(-b)$. En réinjectant ces égalités dans les égalités précédentes et en replaçant le signe moins, on obtient
\[
\lboxed{\cos(a-b)} = \cos(a)\cos(b) - \sin(a)(-\sin(b)) \rboxed{= \cos(a)\cos(b) + \sin(a)\sin(b)}
\]
\[
\lboxed{\sin(a-b)} = \sin(a)\cos(b) + \cos(a)(-\sin(b)) \rboxed{=  \sin(a)\cos(b) - \cos(a)\sin(b)}
\]
\end{proof}

\begin{theorem}[Formule d'addition de tangente]
\label{trigonometrie:addition:tangente}
Soient $a, b \in \setRealNumbers$, alors
\[
\tan(a+b) = \dfrac{\tan(a)+\tan(b)}{1 - \tan(a)\tan(b)}
\qquad\qquad
\tan(a-b) = \dfrac{\tan(a)-\tan(b)}{1 + \tan(a)\tan(b)}
\]
\end{theorem}

\begin{proof}
Soient $a, b \in \setRealNumbers$, alors par définition, on a
\[
\tan(a+b) = \dfrac{\sin(a+b)}{\cos(a+b)}
\]
En appliquant les formules du théorème \ref{trigonometrie:addition}, on a 
\[
\tan(a+b) = \dfrac{\sin(a)\cos(b) + \cos(a)\sin(b)}{\cos(a)\cos(b) - \sin(a)\sin(b)}
\]
En divisant en haut et en bas par $\cos(a)\cos(b)$, on obtient
\[
\tan(a+b) 
=\dfrac{\dfrac{\sin(a)\bcancel{\cos(b)}}{\cos(a)\bcancel{\cos(b)}} + \dfrac{\sin(b)\bcancel{\cos(a)}}{\cos(b)\bcancel{\cos(a)}}}{\dfrac{\bcancel{\cos(a)}\bcancel{\cos(b)}}{\bcancel{\cos(a)}\bcancel{\cos(b)}} - \dfrac{\sin(a)\sin(b)}{\cos(a)\cos(b)}} 
= \dfrac{\dfrac{\sin(a)}{\cos(a)} + \dfrac{\sin(b)}{\cos(b)}}{1 - \dfrac{\sin(a)\sin(b)}{\cos(a)\cos(b)}}
\]
D'où $\boxed{\tan(a+b) = \dfrac{\tan(a)+\tan(b)}{1 - \tan(a)\tan(b)}}$ en reprenant la définition de la tangente. 

D'autre part, en prenant $-b$ à la place de $b$, on obtient
\[
\tan(a-b) = \dfrac{\tan(a)+\tan(-b)}{1 - \tan(a)\tan(-b)}
\]
Comme la fonction tangente est impaire, on a $\tan(-b)=-\tan(b)$, donc
\[
\lboxed{\tan(a-b)} = \dfrac{\tan(a)-\tan(b)}{1 - \tan(a)(-\tan(b))} \rboxed{= \dfrac{\tan(a)-\tan(b)}{1 + \tan(a)\tan(b)}}
\]
\end{proof}