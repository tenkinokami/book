\begin{property}
La famille des polynômes de Tchebychev de première espèce est orthogonale par rapport au produit scalaire
\[
(f, g) \longmapsto \pscal{f}{g} = \displaystyle\int\limits_{0}^{1} \omega(x) f(x) g(x) dx
\]
avec $\omega(x)=\dfrac{1}{\sqrt{1-x^2}}$ une fonction de poids. 
\end{property}

\begin{proof}
Soient $i, j \in \setNaturalNumbers$, alors
\[
\pscal{\polChebCos{i}}{\polChebCos{j}} = \displaystyle\int\limits_{0}^{1} \omega(x) \polChebCos{i}(x)\polChebCos{j}(x) dx
\]
Posons $x = \cos(u)$, alors $dx = \sin(u) du$. Le changement de variable est $C^1$ et bijectif, donc 
\[
\pscal{\polChebCos{i}}{\polChebCos{j}} = \displaystyle\int\limits_{0}^{\pi} \omega(\cos(u))\polChebCos{i}(\cos(u))\polChebCos{j}(x) \sin(u) du
\]
Or on remarque en utilisant l'identité trigonométrique que
\[
\forall u \in \realIntervalCC{0}{\pi}, \:\omega(\cos(u))=\dfrac{1}{\sqrt{1-\cos(u)^2}} = \dfrac{1}{\sqrt{\sin^2(u)}} = \dfrac{1}{\sin(u)}
\]
Car la fonction sinus est positive sur l'intervalle considérée. En réinjectant dans l'expression précédente, on obtient 
\[
\pscal{\polChebCos{i}}{\polChebCos{j}} = \displaystyle\int\limits_{0}^{\pi} \dfrac{1}{\sin(u)}\sin(u)\polChebCos{i}(\cos(u))\polChebCos{j}(x)  du = \displaystyle\int\limits_{0}^{\pi} \polChebCos{i}(\cos(u))\polChebCos{j}(x)  du
\]
Or on sait que 
\[
\polChebCos{i}(\cos(u)) = \cos(iu)
\]
Et que 
\[
\polChebCos{j}(\cos(u)) = \cos(ju)
\]
Donc 
\[
\pscal{\polChebCos{i}}{\polChebCos{j}} 
= \displaystyle\int\limits_{0}^{\pi} \cos(iu) \cos(ju) du
\]
On a donc par théorème
\[
\lboxed{\pscal{\polChebCos{i}}{\polChebCos{j}} }
=  \displaystyle\int\limits_{0}^{\pi} \cos(iu) \cos(ju) du 
\rboxed{= \left\{\begin{array}{l}
0 \quad\text{ si } n \ne m\\
\pi \quad\text{ si } n = m = 0\\
\dfrac{\pi}{2} \quad\text{ sinon}
\end{array}\right.}
\]
\end{proof}