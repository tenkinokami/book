%%% NOTATIONS MATHEMATIQUES : 

%%% Liste des fonctions 
% \cos \sin \tan
% \csc \cot \sec
% \exp \ln \log \lg (log binaire)
% \ker \dim \projlim 
% \limsup \liminf
% \min \max \sup \lim \inf
% \sinh \cosh \coth \tanh
% \arcsin \arctan
% \arg
% \deg
% \gcd
% \det
% \Pr
% \hom
% \mod \bmod \pmod

%% ENSEMBLES DE NOMBRES 
\newcommand{\setNaturalNumbers}{\mathbb{N}}
\newcommand{\setPrimeNumbers}{\wp}
\newcommand{\setIntegers}{\mathbb{Z}}
\newcommand{\setDecimalNumbers}{\mathbb{D}}
\newcommand{\setRationalNumbers}{\mathbb{Q}}
\newcommand{\setRealNumbers}{\mathbb{R}}
\newcommand{\setComplexNumbers}{\mathbb{C}}
\newcommand{\setQuaternion}{\mathbb{H}}
\newcommand{\setOctonion}{\mathbb{O}}
\newcommand{\setAlgebraicNumbers}{\mathbb{A}}
\newcommand{\setConstructibleNumbers}{\mathbb{CO}}
\NewDocumentCommand{\setUnimodularComplexNumbers}{o}{\mathbb{U}\IfValueT{#1}{_{#1}}}

\newcommand{\realIntervalCC}[2]{\left[ #1 ,  #2  \right]}
\newcommand{\realIntervalCO}[2]{\left[  #1  ,  #2 \right[} 
\newcommand{\realIntervalOC}[2]{\left]  #1  ,  #2 \right]} 
\newcommand{\realIntervalOO}[2]{\left] #1 , #2 \right[}
\WithSuffix\newcommand\realIntervalCC*[2]{[ #1 , #2 ]}
\WithSuffix\newcommand\realIntervalCO*[2]{[ #1, #2[}
\WithSuffix\newcommand\realIntervalOC*[2]{] #1 ,#2 ]}
\WithSuffix\newcommand\realIntervalOO*[2]{] #1, #2[}

\newcommand{\integerIntervalCC}[2]{\left\llbracket \, #1 \, , \, #2 \, \right\rrbracket}
\newcommand{\integerIntervalCO}[2]{\left\llbracket \, #1 \, , \, #2 \, \right\llbracket} 
\newcommand{\integerIntervalOC}[2]{\left\rrbracket \, #1 \, , \, #2 \, \right\rrbracket} 
\newcommand{\integerIntervalOO}[2]{\left\rrbracket \, #1 \, , \, #2 \, \right\llbracket}
\WithSuffix\newcommand\integerIntervalCC*[2]{\llbracket \, #1 \, , \, #2 \, \rrbracket}
\WithSuffix\newcommand\integerIntervalCO*[2]{\llbracket \, #1 \, , \, #2 \, \llbracket}
\WithSuffix\newcommand\integerIntervalOC*[2]{\rrbracket \, #1 \, , \, #2 \, \rrbracket}
\WithSuffix\newcommand\integerIntervalOO*[2]{\rrbracket \, #1 \, , \, #2 \, \llbracket}

%% ENSEMBLES DE FONCTIONS : 
\newcommand{\setFunctions}[2]{\mathcal{F}(#1,#2)}
\newcommand{\setContinuousFunctions}[3]{\mathscr{C}^{#1}(#2,#3)}
\newcommand{\setDifferentiableFunctions}[3]{{\Delta}^{#1}(#2,#3)}
\NewDocumentCommand{\setPeriodicFunctions}{o m m}{T\IfValueT{#1}{_{#1}}(#2,#3)}
\newcommand{\setBoundedFunctions}[2]{{B}(#1,#2)}
\newcommand{\setConvexFunctions}[2]{\textrm{Conv}(#1,#2)}
\NewDocumentCommand{\setPolynomials}{m o m}{{#1}\IfValueT{#2}{_{#2}}[#3]}

%% ENSEMBLES D’ALGÈBRE LINEAIRE : 
\newcommand{\setLinearSpaces}{\mathbb{LS}}
\newcommand{\setLinearMap}[2]{\mathscr{L}(#1,#2)}
\newcommand{\setEndomorphisms}[1]{\mathscr{L}(#1)}
\newcommand{\setAutomorphisms}[1]{\mathscr{GL}(#1)}
\newcommand{\setIsomorphisms}[2]{\textrm{Isom}(#1,#2)}
\newcommand{\setHomeomorphisms}[2]{\textrm{Hom}(#1,#2)}
\newcommand{\setDiffeomorphisms}[2]{\textrm{Diff}(#1,#2)}

%% FONCTION USUELLES : 
\let\log\oldlog
\NewDocumentCommand{\log}{o}{\oldlog\IfValueT{#1}{_{#1}}}

\DeclareMathOperator{\Modulus}{mod}
\newcommand{\modulus}[1]{\Modulus\left(#1\right)}%{\left\lvert #1 \right\rvert}
\WithSuffix\newcommand\modulus*[1]{\Modulus\,(#1)}%{\lvert #1 \rvert}

%\newcommand{\abs}{abs}
%\DeclareMathOperator{\Re}{Re}
%\DeclareMathOperator{\Im}{Im}

\DeclareMathOperator*{\Infimum}{inf}
\DeclareMathOperator*{\Supremum}{sup}
\NewDocumentCommand{\infimum}{o o m}{\Infimum\limits_{\IfValueT{#1}{#1 \in #2}}\left\{\,#3\,\right\}}
\NewDocumentCommand{\supremum}{o o m}{\Supremum\limits_{\IfValueT{#1}{#1 \in #2}}\left\{\,#3\,\right\}}
\WithSuffix\NewDocumentCommand\infimum*{o o m}{\Infimum\limits_{\IfValueT{#1}{#1\in #2}}\{\,#3\,\}}
\WithSuffix\NewDocumentCommand\supremum*{o o m}{\Supremum\limits_{\IfValueT{#1}{#1 \in #2}}\{\,#3\,\}}

%\NewDocumentCommand{\norm}{o m}{\left\|\, #2 \,\right\|_{\IfValueT{#1}{#1}}}
%\WithSuffix\NewDocumentCommand\norm*{o m}{\|\, #2 \,\|_{\IfValueT{#1}{#1}}}

%% PROBABILITE : 
\DeclareMathOperator{\Probability}{\mathbb{P}}
\DeclareMathOperator{\ExpectedValue}{\mathbb{E}}
\DeclareMathOperator{\Variance}{\mathbb{V}}
\DeclareMathOperator{\Covariance}{\textrm{Cov}}
\DeclareMathOperator{\StandardDeviation}{\sigma}
%\newcommand{\meanAbsoluteDeviation}{}
\newcommand{\probability}[1]{\Probability\left(#1\right)}
\newcommand{\expectedValue}[1]{\ExpectedValue\left[#1\right]}
\newcommand{\variance}[1]{\Variance\left[#1\right]}
\newcommand{\covariance}[2]{\Covariance\left(#1, \, #2\right)}
\newcommand{\standardDeviation}[1]{\StandardDeviation\left[#1\right]}
\WithSuffix\newcommand\probability*[1]{\Probability\,(#1)}
\WithSuffix\newcommand\expectedValue*[1]{\ExpectedValue\, [#1]}
\WithSuffix\newcommand\variance*[1]{\Variance\, [#1]}
\WithSuffix\newcommand\standardDeviation*[1]{\StandardDeviation\, [#1]}
\WithSuffix\newcommand\covariance*[2]{\Covariance \, (#1, \, #2)}

%% ARITHMETIQUE : 
%\newcommand{\ceil}[1]{\left\lceil \, #1  \, \right\rceil}
%\newcommand{\floor}[1]{\left\lfloor \, #1 \, \right\rfloor}
%\WithSuffix\newcommand\ceil*[2]{\lceil \, #1 \, \rceil}
%\WithSuffix\newcommand\floor*[2]{\lfloor \, #1 \, \rfloor}

\DeclareMathOperator{\Cardinality}{card}
\newcommand{\cardinality}[1]{\Cardinality\left(#1\right)}
\WithSuffix\newcommand\cardinality*[1]{\Cardinality\,(#1)}
% \let\gcd\temp
% \DeclareMathOperator{\Lcm}{ppcm}
% \DeclareMathOperator{\Gcd}{pgcd}
% \newcommand{\lcm}[2]{\Lcm\left(#1, \, #2\right)}
% \newcommand{\gcd}[2]{\Gcd\left(#1, \, #2\right)}
% \WithSuffix\newcommand\lcm*[2]{\Lcm\,(#1, \, #2)}
% \WithSuffix\newcommand\gcd*[2]{\Gcd\,(#1, \, #2)}

\NewDocumentCommand{\set}{m o}{\left\{#1\IfValueT{#2}{\:\:\dashleftarrow\:\:#2}\,\right\}}
\WithSuffix\NewDocumentCommand\set*{m o}{\{\,#1\IfValueT{#2}{\:\:\dashleftarrow\:\:#2}\,\}}


%% LOGIQUE :
\newcommand{\as}[1]{\left/ #1\right.}

\NewDocumentCommand{\function}{o o o m m}%
{%
\IfValueT{#1}{#1 :}% Nom de la fonction 
	\IfValueTF{#2}{%
    \left|
	\begin{array}{l c l}%
	#2 & \longrightarrow & #3\\ % Ensemble de départ et d'arrivée
	#4 & \longmapsto & #5\\ % Valeurs de la fonction
	\end{array}\right.%
	}%
    {%
	#4 \longmapsto #5% Valeurs de la fonction
	}%
}