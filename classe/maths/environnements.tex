\newcounter{nbtheorem}[chapter]
\setcounter{nbtheorem}{0}
\definecolor{or}{RGB}{255,215,0}

\let\theorem\temp
\NewDocumentEnvironment{theorem}{o}
{\refstepcounter{nbtheorem}%
	\begin{bclogo}[%
		nobreak,%
		ombre = false,%
		arrondi = 0.2,%
		logo = \hspace{17pt},%
		barre = none,%
        couleurBarre = red!5,%
		couleur = red!5,%
        couleurBord = black!40!colsection,%
        epBord = 0.7]%
{\color{black!30!colsection}\hspace{-15pt}Théorème \thenbtheorem \IfValueT{#1}{: #1}}}%
{\end{bclogo}}%

\NewDocumentEnvironment{lemma}{o}
{\refstepcounter{nbtheorem}%
	\begin{bclogo}[%
		nobreak,%
		ombre = false,%
		arrondi = 0.2,%
		logo = \hspace{17pt},%
		barre = none,%
        couleurBarre = yellow!5,%
		couleur = yellow!5,%
        couleurBord = black!40!colsoussoussection,%
        epBord = 0.7]%
{\color{black!30!colsoussoussection}\hspace{-15pt}Lemme \thenbtheorem \IfValueT{#1}{: #1}}}%
{\end{bclogo}}%

\NewDocumentEnvironment{corollary}{o}
{\refstepcounter{nbtheorem}%
	\begin{bclogo}[%
		nobreak,%
		ombre = false,%
		arrondi = 0.2,%
		logo = \hspace{17pt},%
		barre = none,%
        couleurBarre = yellow!5,%
		couleur = yellow!5,%
        couleurBord = black!40!colsoussoussection,%
        epBord = 0.7]%
{\color{black!40!colsoussoussection}\hspace{-15pt}Corollaire \thenbtheorem \IfValueT{#1}{: #1}}}%
{\end{bclogo}}%

\NewDocumentEnvironment{proposition}{o}
{\refstepcounter{nbtheorem}%
	\begin{bclogo}[%
		nobreak,%
		ombre = false,%
		arrondi = 0.2,%
		logo = \hspace{17pt},%
		barre = none,%
        couleurBarre = blue!5,%
		couleur = blue!5,%
        couleurBord = black!40!colsoussection,%
        epBord = 0.7]%
{\color{black!30!colsoussection}\hspace{-15pt}Proposition \thenbtheorem \IfValueT{#1}{: #1}}}%
{\end{bclogo}}%

\NewDocumentEnvironment{property}{o}
{\refstepcounter{nbtheorem}%
	\begin{bclogo}[%
		nobreak,%
		ombre = false,%
		arrondi = 0.2,%
		logo = \hspace{17pt},%
		barre = none,%
        couleurBarre = blue!5,%
		couleur = blue!5,%
        couleurBord = black!40!colsoussection,%
        epBord = 0.7]%
{\color{black!30!colsoussection}\hspace{-15pt}Propriété \thenbtheorem \IfValueT{#1}{: #1}}}%
{\end{bclogo}}%

\definecolor{darkgreen}{RGB}{0,100,0}
\NewDocumentEnvironment{definition}{o}
{\begin{bclogo}[nobreak,%
					ombre = false,%
					arrondi = 0.2,%
					logo = \hspace{17pt},%
					barre = none,%
                    couleurBarre = green!5,%
					couleur = green!5,%
                    couleurBord = black!30!darkgreen,%
                    epBord = 0.7]%
{\color{black!30!darkgreen}\hfil\:\:#1\:\:}%
}%
{\end{bclogo}}%

\newcounter{nbexercice}[chapter]
\setcounter{nbexercice}{0}

\definecolor{palecyan}{RGB}{175, 238, 238}
\NewDocumentEnvironment{exercice}{o}
{\refstepcounter{nbexercice}%
\begin{bclogo}[nobreak,%
			   	ombre = false,%
			   	arrondi = 0.2,%
				logo = \hspace{17pt},%
				barre = none,%
                couleurBarre = cyan!5,%
				couleur = cyan!5,%
                couleurBord = black!30!palecyan,%
                epBord = 0.7]%
{\color{black!60!palecyan}\hspace{-15pt}Exercice \thenbexercice \IfValueT{#1}{: #1}}}%
{\end{bclogo}}%

\theoremstyle{definition}
\newcounter{nbexample}[chapter]
\setcounter{nbexample}{0}

\newtheorem{example}[nbexample]{Exemple}