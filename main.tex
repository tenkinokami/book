\documentclass[mathmodern,openany,11pt]{livre}
%%% NOTATIONS MATHEMATIQUES : 

%%% Liste des fonctions 
% \cos \sin \tan
% \csc \cot \sec
% \exp \ln \log \lg (log binaire)
% \ker \dim \projlim 
% \limsup \liminf
% \min \max \sup \lim \inf
% \sinh \cosh \coth \tanh
% \arcsin \arctan
% \arg
% \deg
% \gcd
% \det
% \Pr
% \hom
% \mod \bmod \pmod

%% ENSEMBLES DE NOMBRES 
\newcommand{\setNaturalNumbers}{\mathbb{N}}
\newcommand{\setPrimeNumbers}{\wp}
\newcommand{\setIntegers}{\mathbb{Z}}
\newcommand{\setDecimalNumbers}{\mathbb{D}}
\newcommand{\setRationalNumbers}{\mathbb{Q}}
\newcommand{\setRealNumbers}{\mathbb{R}}
\newcommand{\setComplexNumbers}{\mathbb{C}}
\newcommand{\setQuaternion}{\mathbb{H}}
\newcommand{\setOctonion}{\mathbb{O}}
\newcommand{\setAlgebraicNumbers}{\mathbb{A}}
\newcommand{\setConstructibleNumbers}{\mathbb{CO}}
\NewDocumentCommand{\setUnimodularComplexNumbers}{o}{\mathbb{U}\IfValueT{#1}{_{#1}}}

\newcommand{\realIntervalCC}[2]{\left[ #1 ,  #2  \right]}
\newcommand{\realIntervalCO}[2]{\left[  #1  ,  #2 \right[} 
\newcommand{\realIntervalOC}[2]{\left]  #1  ,  #2 \right]} 
\newcommand{\realIntervalOO}[2]{\left] #1 , #2 \right[}
\WithSuffix\newcommand\realIntervalCC*[2]{[ #1 , #2 ]}
\WithSuffix\newcommand\realIntervalCO*[2]{[ #1, #2[}
\WithSuffix\newcommand\realIntervalOC*[2]{] #1 ,#2 ]}
\WithSuffix\newcommand\realIntervalOO*[2]{] #1, #2[}

\newcommand{\integerIntervalCC}[2]{\left\llbracket \, #1 \, , \, #2 \, \right\rrbracket}
\newcommand{\integerIntervalCO}[2]{\left\llbracket \, #1 \, , \, #2 \, \right\llbracket} 
\newcommand{\integerIntervalOC}[2]{\left\rrbracket \, #1 \, , \, #2 \, \right\rrbracket} 
\newcommand{\integerIntervalOO}[2]{\left\rrbracket \, #1 \, , \, #2 \, \right\llbracket}
\WithSuffix\newcommand\integerIntervalCC*[2]{\llbracket \, #1 \, , \, #2 \, \rrbracket}
\WithSuffix\newcommand\integerIntervalCO*[2]{\llbracket \, #1 \, , \, #2 \, \llbracket}
\WithSuffix\newcommand\integerIntervalOC*[2]{\rrbracket \, #1 \, , \, #2 \, \rrbracket}
\WithSuffix\newcommand\integerIntervalOO*[2]{\rrbracket \, #1 \, , \, #2 \, \llbracket}

%% ENSEMBLES DE FONCTIONS : 
\newcommand{\setFunctions}[2]{\mathcal{F}\,(#1,\, #2)}
\newcommand{\setContinuousFunctions}[3]{\mathscr{C}^{ #1}\,(#2,\, #3)}
\newcommand{\setDifferentiableFunctions}[3]{{\Delta}^{ #1}\,(#2,\, #3)}
\NewDocumentCommand{\setPeriodicFunctions}{o m m}{{T}_{\IfValueT{#1}{ #1}}\,(#2,\, #3)}
\newcommand{\setBoundedFunctions}[2]{{B}\,(#1,\, #2)}
\newcommand{\setConvexFunctions}[2]{\textrm{Conv}\,(#1,\, #2)}
\NewDocumentCommand{\setPolynomials}{m o m}{{#1}\IfValueT{#2}{_{#2}}[#3]}

%% ENSEMBLES D’ALGÈBRE LINEAIRE : 
\newcommand{\setLinearSpaces}{\mathbb{LS}}
\newcommand{\setLinearMap}[2]{\mathscr{L}\,(#1,\, #2)}
\newcommand{\setEndomorphisms}[1]{\mathscr{L}\,(#1)}
\newcommand{\setAutomorphisms}[1]{\mathscr{GL}\,(#1)}
\newcommand{\setIsomorphisms}[2]{\textrm{Isom}\,(#1,\, #2)}
\newcommand{\setHomeomorphisms}[2]{\textrm{Hom}\,(#1,\, #2)}
\newcommand{\setDiffeomorphisms}[2]{\textrm{Diff}\,(#1,\, #2)}

%% FONCTION USUELLES : 
\DeclareMathOperator{\Cosine}{cos}
\DeclareMathOperator{\Sine}{sin}
\DeclareMathOperator{\Tangent}{tan}
\NewDocumentCommand{\cosine}{o m}{\Cosine\IfValueTF{#1}{^{#1}}\left(#2\right)}
\NewDocumentCommand{\sine}{o m}{\Sine\IfValueTF{#1}{^{#1}}\left(#2\right)}
\NewDocumentCommand{\tangent}{o m}{\Tangent\IfValueTF{#1}{^{#1}}\left(#2\right)}
\WithSuffix\NewDocumentCommand\sine*{o m}{\Sine\IfValueTF{#1}{^{#1}}\,(#2)}
\WithSuffix\NewDocumentCommand\cosine*{o m}{\Cosine\IfValueTF{#1}{^{#1}}\,(#2)}
\WithSuffix\NewDocumentCommand\tangent*{o m}{\Tangent\IfValueTF{#1}{^{#1}}\,(#2)}

\DeclareMathOperator{\CosineHyperbolic}{cosh}
\DeclareMathOperator{\SineHyperbolic}{sinh}
\DeclareMathOperator{\TangentHyperbolic}{tanh}
\NewDocumentCommand{\cosineHyperbolic}{o m}{\CosineHyperbolic\IfValueT{#1}{^{#1}}\left(#2\right)}
\NewDocumentCommand{\sineHyperbolic}{o m}{\SineHyperbolic\IfValueT{#1}{^{#1}}\left(#2\right)}
\NewDocumentCommand{\tangentHyperbolic}{o m}{\TangentHyperbolic\IfValueT{#1}{^{#1}}\left(#2\right)}
\WithSuffix\NewDocumentCommand\sineHyperbolic*{o m}{\SineHyperbolic\IfValueT{#1}{^{#1}}\,(#2)}
\WithSuffix\NewDocumentCommand\cosineHyperbolic*{o m}{\CosineHyperbolic\IfValueT{#1}{^{#1}}\,(#2)}
\WithSuffix\NewDocumentCommand\tangentHyperbolic*{o m}{\TangentHyperbolic\IfValueT{#1}{^{#1}}\,(#2)}

\DeclareMathOperator{\CosineHyperbolicInverse}{argcosh}
\DeclareMathOperator{\SineHyperbolicInverse}{argsinh}
\DeclareMathOperator{\TangentHyperbolicInverse}{argtanh}
\NewDocumentCommand{\cosineHyperbolicInverse}{o m}{\CosineHyperbolicInverse\IfValueT{#1}{^{#1}}\left(#2\right)}
\NewDocumentCommand{\sineHyperbolicInverse}{o m}{\SineHyperbolicInverse\IfValueT{#1}{^{#1}}\left(#2\right)}
\NewDocumentCommand{\tangentHyperbolicInverse}{o m}{\TangentHyperbolicInverse\IfValueT{#1}{^{#1}}\left(#2\right)}
\WithSuffix\NewDocumentCommand\sineHyperbolicInverse*{o m}{\SineHyperbolicInverse\IfValueT{#1}{^{#1}}\,(#2)}
\WithSuffix\NewDocumentCommand\cosineHyperbolicInverse*{o m}{\CosineHyperbolicInverse\IfValueT{#1}{^{#1}}\,(#2)}
\WithSuffix\NewDocumentCommand\tangentHyperbolicInverse*{o m}{\TangentHyperbolicInverse\IfValueT{#1}{^{#1}}\,(#2)}

\DeclareMathOperator{\CosineInverse}{arccos}
\DeclareMathOperator{\SineInverse}{arcsin}
\DeclareMathOperator{\TangentInverse}{arctan}
\NewDocumentCommand{\cosineInverse}{o m}{\CosineInverse\IfValueT{#1}{^{#1}}\left(#2\right)}
\NewDocumentCommand{\sineInverse}{o m}{\SineInverse\IfValueT{#1}{^{#1}}\left(#2\right)}
\NewDocumentCommand{\tangentInverse}{o m}{\TangentInverse\IfValueT{#1}{^{#1}}\left(#2\right)}
\WithSuffix\NewDocumentCommand\sineInverse*{o m}{\SineInverse\IfValueT{#1}{^{#1}}\,(#2)}
\WithSuffix\NewDocumentCommand\cosineInverse*{o m}{\CosinecInverse\IfValueT{#1}{^{#1}}\,(#2)}
\WithSuffix\NewDocumentCommand\tangentInverse*{o m}{\TangentInverse\IfValueT{#1}{^{#1}}\,(#2)}

\DeclareMathOperator{\Cotangent}{cot}
\DeclareMathOperator{\Secant}{sec}
\DeclareMathOperator{\Cosecant}{csc}
\NewDocumentCommand{\cotangent}{o m}{\Cotangent\IfValueT{#1}{^{#1}}\left(#2\right)}
\NewDocumentCommand{\secant}{o m}{\Secant\IfValueT{#1}{^{#1}}\left(#2\right)}
\NewDocumentCommand{\cosecant}{o m}{\Cosecant\IfValueT{#1}{^{#1}}\left(#2\right)}
\WithSuffix\NewDocumentCommand\cotangent*{o m}{\Cotangent\IfValueT{#1}{^{#1}}\,(#2)}
\WithSuffix\NewDocumentCommand\secant*{o m}{\Secant\IfValueT{#1}{^{#1}}\,(#2)}
\WithSuffix\NewDocumentCommand\cosecant*{o m}{\Cosecant\IfValueT{#1}{^{#1}}\,(#2)}

\DeclareMathOperator{\Exponential}{exp}
\DeclareMathOperator{\LogarithmNatural}{ln}
\DeclareMathOperator{\Logarithm}{log}
\NewDocumentCommand{\exponential}{o m}{\Exponential\IfValueT{#1}{^{#1}}\left(#2\right)}
\NewDocumentCommand{\logarithmNatural}{o m}{\LogarithmNatural\IfValueT{#1}{^{#1}}\left(#2\right)}
\NewDocumentCommand{\logarithm}{o o m}{\Logarithm\IfValueTF{#1}{_{#1}}{_{10}}\IfValueTF{#2}{^{#2}}\left(#3\right)}
\WithSuffix\NewDocumentCommand\exponential*{o m}{\Exponential\IfValueT{#1}{^{#1}}\,(#2)}
\WithSuffix\NewDocumentCommand\logarithmNatural*{o m}{\LogarithmNatural\IfValueT{#1}{^{#1}}\,(#2)}
\WithSuffix\NewDocumentCommand\logarithm*{o o m}{\Logarithm_{\IfValueTF{#1}{#1}{10}}\IfValueT{#2}{^{#2}}\,(#3)}

\DeclareMathOperator{\Modulus}{mod}
\newcommand{\modulus}[1]{\Modulus\left(#1\right)}%{\left\lvert #1 \right\rvert}
\WithSuffix\newcommand\modulus*[1]{\Modulus\,(#1)}%{\lvert #1 \rvert}

\newcommand{\powerProduct}[1]{\overset{\times}{#1}}
\newcommand{\powerComposition}[1]{\overset{\circ}{#1}}

\DeclareMathOperator{\AbsoluteValue}{abs}
\DeclareMathOperator{\RealPart}{Re}
\DeclareMathOperator{\ImaginaryPart}{Im}
\DeclareMathOperator{\Argument}{arg}
\newcommand{\absoluteValue}[1]{\AbsoluteValue\left(#1\right)}
\newcommand{\realPart}[1]{\RealPart\left(#1\right)}
\newcommand{\imaginaryPart}[1]{\ImaginaryPart\left(#1\right)}
\newcommand{\argument}[1]{\Argument\left(#1\right)}
\WithSuffix\newcommand\absoluteValue*[1]{\AbsoluteValue\,(#1)}
\WithSuffix\newcommand\realPart*[1]{\RealPart\,(#1)}
\WithSuffix\newcommand\imaginaryPart*[1]{\ImaginaryPartPart\,(#1)}
\WithSuffix\newcommand\argument*[1]{\Argument(#1)}

\DeclareMathOperator*{\Infimum}{inf}
\DeclareMathOperator*{\Supremum}{sup}
\NewDocumentCommand{\infimum}{o o m}{\Infimum\limits_{\IfValueT{#1}{#1 \in #2}}\left\{\,#3\,\right\}}
\NewDocumentCommand{\supremum}{o o m}{\Supremum\limits_{\IfValueT{#1}{#1 \in #2}}\left\{\,#3\,\right\}}
\WithSuffix\NewDocumentCommand\infimum*{o o m}{\Infimum\limits_{\IfValueT{#1}{#1\in #2}}\{\,#3\,\}}
\WithSuffix\NewDocumentCommand\supremum*{o o m}{\Supremum\limits_{\IfValueT{#1}{#1 \in #2}}\{\,#3\,\}}

\NewDocumentCommand{\norm}{o m}{\left\|\, #2 \,\right\|_{\IfValueT{#1}{#1}}}
\WithSuffix\NewDocumentCommand\norm*{o m}{\|\, #2 \,\|_{\IfValueT{#1}{#1}}}

%% PROBABILITE : 
\DeclareMathOperator{\Probability}{\mathbb{P}}
\DeclareMathOperator{\ExpectedValue}{\mathbb{E}}
\DeclareMathOperator{\Variance}{\mathbb{V}}
\DeclareMathOperator{\Covariance}{\textrm{Cov}}
\DeclareMathOperator{\StandardDeviation}{\sigma}
%\newcommand{\meanAbsoluteDeviation}{}
\newcommand{\probability}[1]{\Probability\left(#1\right)}
\newcommand{\expectedValue}[1]{\ExpectedValue\left[#1\right]}
\newcommand{\variance}[1]{\Variance\left[#1\right]}
\newcommand{\covariance}[2]{\Covariance\left(#1, \, #2\right)}
\newcommand{\standardDeviation}[1]{\StandardDeviation\left[#1\right]}
\WithSuffix\newcommand\probability*[1]{\Probability\,(#1)}
\WithSuffix\newcommand\expectedValue*[1]{\ExpectedValue\, [#1]}
\WithSuffix\newcommand\variance*[1]{\Variance\, [#1]}
\WithSuffix\newcommand\standardDeviation*[1]{\StandardDeviation\, [#1]}
\WithSuffix\newcommand\covariance*[2]{\Covariance \, (#1, \, #2)}

%% ARITHMETIQUE : 
\newcommand{\ceil}[1]{\left\lceil \, #1  \, \right\rceil}
\newcommand{\floor}[1]{\left\lfloor \, #1 \, \right\rfloor}
\WithSuffix\newcommand\ceil*[2]{\lceil \, #1 \, \rceil}
\WithSuffix\newcommand\floor*[2]{\lfloor \, #1 \, \rfloor}

\DeclareMathOperator{\Cardinality}{card}
\newcommand{\cardinality}[1]{\Cardinality\left(#1\right)}
\WithSuffix\newcommand\cardinality*[1]{\Cardinality\,(#1)}

% \let\gcd\temp
% \DeclareMathOperator{\Lcm}{ppcm}
% \DeclareMathOperator{\Gcd}{pgcd}
% \newcommand{\lcm}[2]{\Lcm\left(#1, \, #2\right)}
% \newcommand{\gcd}[2]{\Gcd\left(#1, \, #2\right)}
% \WithSuffix\newcommand\lcm*[2]{\Lcm\,(#1, \, #2)}
% \WithSuffix\newcommand\gcd*[2]{\Gcd\,(#1, \, #2)}

\NewDocumentCommand{\set}{m o}{\left\{\,#1\IfValueT{#2}{\:\:\dashleftarrow\:\:#2}\,\right\}}
\WithSuffix\NewDocumentCommand\set*{m o}{\{\,#1\IfValueT{#2}{\:\:\dashleftarrow\:\:#2}\,\}}

\DeclareMathOperator*{\MinimaLocal}{min}
\DeclareMathOperator*{\MaximaLocal}{max}
\NewDocumentCommand{\minimaLocal}{o o m}{\MinimaLocal\limits_{\IfValueT{#1}{#1 \in #2}}^{L}\left\{\,#3\,\right\}}
\NewDocumentCommand{\maximaLocal}{o o m}{\MaximaLocal\limits_{\IfValueT{#1}{#1 \in #2}}^{L}\left\{\,#3\,\right\}}
\WithSuffix\NewDocumentCommand\minimaLocal*{o o m}{\MinimaLocal\limits_{\IfValueT{#1}{#1\in #2}}^{L}\{\,#3\,\}}
\WithSuffix\NewDocumentCommand\maximaLocal*{o o m}{\MaximaLocal\limits_{\IfValueT{#1}{#1 \in #2}}^{L}\{\,#3\,\}}

\DeclareMathOperator*{\ArgumentsOfTheMinimaLocal}{argmin}
\DeclareMathOperator*{\ArgumentsOfTheMaximaLocal}{argmax}
\NewDocumentCommand{\argumentsOfTheMinimaLocal}{o o m}{\ArgumentsOfTheMinimaLocal\limits_{\IfValueT{#1}{#1 \in #2}}^{L}\left\{\,#3\,\right\}}
\NewDocumentCommand{\argumentsOfTheMaximaLocal}{o o m}{\ArgumentsOfTheMaximaLocal\limits_{\IfValueT{#1}{#1 \in #2}}^{L}\left\{\,#3\,\right\}}
\WithSuffix\NewDocumentCommand\argumentsOfTheMinimaLocal*{o o m}{\ArgumentsOfTheMinimaLocal\limits_{\IfValueT{#1}{#1 \in #2}}^{L}\{\,#3\,\}}
\WithSuffix\NewDocumentCommand\argumentsOfTheMaximaLocal*{o o m}{\ArgumentsOfTheMaximaLocal\limits_{\IfValueT{#1}{#1 \in #2}}^{L}\{\,#3\,\}}

\DeclareMathOperator*{\MinimumGlobal}{min}
\DeclareMathOperator*{\MaximumGlobal}{max}%\text{\begin{japanesefont}高\end{japanesefont}}
\NewDocumentCommand{\minimumGlobal}{o o m}{\MinimumGlobal\limits_{\IfValueT{#1}{#1 \in \footnotesize\vcenter{\hbox{$#2$}}}}\left\{\,#3\,\right\}}
\NewDocumentCommand{\maximumGlobal}{o o m}{\MaximumGlobal\limits_{\IfValueT{#1}{#1 \in \footnotesize\vcenter{\hbox{$#2$}}}}\left\{\,#3\,\right\}}
\WithSuffix\NewDocumentCommand\minimumGlobal*{o o m}{\MinimumGlobal\limits_{\IfValueT{#1}{#1 \in \footnotesize\vcenter{\hbox{$#2$}}}}\{\,#3\,\}}
\WithSuffix\NewDocumentCommand\maximumGlobal*{o o m}{\MaximumGlobal\limits_{\IfValueT{#1}{#1 \in \footnotesize\vcenter{\hbox{$#2$}}}}\{\,#3\,\}}

\DeclareMathOperator*{\ArgumentsOfTheMinimumGlobal}{argmin}
\DeclareMathOperator*{\ArgumentsOfTheMaximumGlobal}{argmax}
\NewDocumentCommand{\argumentsOfTheMinimumGlobal}{o o m}{\ArgumentsOfTheMinimumGlobal\limits_{\IfValueT{#1}{#1 \in\vcenter{\hbox{$#2$}}}}\left\{\,#3\,\right\}}
\NewDocumentCommand{\argumentsOfTheMaximumGlobal}{o o m}{\ArgumentsOfTheMaximumGlobal\limits_{\IfValueT{#1}{#1 \in \vcenter{\hbox{$#2$}}}}\left\{\,#3\,\right\}}
\WithSuffix\NewDocumentCommand\argumentsOfTheMinimumGlobal*{o o m}{\ArgumentsOfTheMinimumGlobal\limits_{\IfValueT{#1}{#1 \in\vcenter{\hbox{$#2$}}}}\{\,#3\,\}}
\WithSuffix\NewDocumentCommand\argumentsOfTheMaximumGlobal*{o o m}{\ArgumentsOfTheMaximimGlobal\limits_{\IfValueT{#1}{#1 \in\vcenter{\hbox{$#2$}}}}\{\,#3\,\}}



%% LOGIQUE :
\newcommand{\existsInAs}[3]{\exists\: #1 \in #2 \:\LARGE\left/\: #3\right.}
\newcommand{\existsUniqueInAs}[3]{\exists\:!\: #1 \in #2 \:\LARGE\left/\: #3\right.}
\newcommand{\forallInProperty}[3]{\forall\: #1 \in #2, \: #3}

\NewDocumentCommand{\function}{o o o m m}%
{%
\IfValueT{#1}{#1 :}% Nom de la fonction 
	\IfValueTF{#2}{%
    \left|
	\begin{array}{l c l}%
	#2 & \longrightarrow & #3\\ % Ensemble de départ et d'arrivée
	#4 & \longmapsto & #5\\ % Valeurs de la fonction
	\end{array}\right.%
	}%
    {%
	#4 \longmapsto #5% Valeurs de la fonction
	}%
}
\allowdisplaybreaks
\raggedbottom

\hypersetup%
{%
pdftitle={Théorie générale des mathématiques},%
pdfauthor={Adrien MEILAC},%
pdfsubject={Mathématiques},%
pdflang={fr}%
}

\newcommand{\lboxed}[1]{{\color{blue}\begin{array}{|l}\hline\\[-2.4ex]{\color{black}#1}\\[1.2ex]\hline\end{array}}}
\newcommand{\rboxed}[1]{{\color{blue}\begin{array}{r|}\hline\\[-2.4ex]{\color{black}#1}\\[1.2ex]\hline\end{array}}}
\newcommand{\kronecker}[2]{\delta_{(#1, #2)}}
\newcommand{\polChebCos}[1]{\mathcal{T}_{#1}}
\newcommand{\polChebSin}[1]{\mathcal{U}_{#1}}

\newcommand{\lrboxed}[1]{{\color{blue}\boxed{\color{black}#1}}}
\newcommand{\pscal}[2]{\left\langle#1, #2\right\rangle}

\begin{document}
%\begin{titlepage}
%\definecolor{PDGcolor}{RGB}{95,158,160}
%\pagecolor{PDGcolor!30}

\begin{center}
\pagecolor{cyan!30}

\vspace*{0.18\textheight}

\rule{0.8\linewidth}{0.7mm}\\[1.5ex]
\rule{\linewidth}{0.7mm}
\vspace{1cm}

{ \Huge \scshape Théorie générale des}\\[4ex]
{ \Huge \scshape mathématiques }
\vspace{1cm}

\rule{\linewidth}{0.7mm}\\[1.5ex]
\rule{0.8\linewidth}{0.7mm}

\vfill

\end{center}
\end{titlepage}
\clearpage
\restoregeometry
\pagecolor{white}
%
%\setlength{\parskip}{0.2\baselineskip}
%
%\tableofcontents
%\newpage
%
%\setlength{\parskip}{0.5\baselineskip}

\pagestyle{default}

\chapter{Trigonométrie}


\begin{lemma}
Soit $f$ une fonction dérivable au voisinage de $0$ telle que $f'(x)= o_{0}(x^n)$ alors $f(x)=o_o(x^{n+1})$
\end{lemma}

\begin{proof}
Soit $\epsilon > 0$, alors 
\[
\exists \alpha > 0 \left/ \forall x \in \realIntervalOO{-\alpha}{\alpha}, |f'(x)|\le \varepsilon |x^n|\right.
\]
Soit $x \in \realIntervalCC{0}{\alpha}$, on sait que 
\[
f(x)-f(0) = \displaystyle\int\limits_{0}^{x} f'(t) dt
\]
Donc par inégalité triangulaire :
\[
|f(x)-f(0)| \le \displaystyle\int\limits_{0}^{x} |f'(t)| dt
\]
Comme $x \in \realIntervalOO{0}{\alpha}$, on a $x < \alpha$ et $\realIntervalCC{0}{x} \subset \realIntervalOO{0}{\alpha}$, on peut donc appliquer l'hypothèse de domination en tout point de l'intégrale :
\[
\displaystyle\int\limits_{0}^{x} |f'(t)| dt \le \displaystyle\int\limits_{0}^{x} \varepsilon |t^n| dt = \epsilon \displaystyle\int\limits_{0}^{x} |t^n| dt
\]
Comme $\function{t}{t^n}$ est positive sur $\realIntervalCC{0}{t}$, on peut enlever les valeurs absolues. on a alors 
\[
\varepsilon \displaystyle\int\limits_{0}^{x} t^n dt 
=
\varepsilon \left[\dfrac{t^{n+1}}{n+1}\right]_{0}^{x} 
= 
\varepsilon \Bigl(\dfrac{x^{n+1}}{n+1} - \underbrace{\dfrac{0^{n+1}}{n+1}}_{=0}\Bigr)
=
\dfrac{\varepsilon}{n+1} x^{n+1}
\]
Comme $n > 1$, on a ainsi $n+1 > 2$ d'où $\dfrac{1}{n+1} \le 1$ donc finalement 
\[
\forall x \in \realIntervalCO{0}{\alpha}, 
|f(x)-f(0)| \le \dfrac{\varepsilon}{n+1} x^{n+1} 
\le 
\varepsilon x^{n+1}
\]

Si $x \in \realIntervalOC{-\alpha}{0}$, alors on peut appliquer l'inégalité triangulaire en inversant les bornes
\[
|f(x)-f(0)| 
\le 
\int\limits_{x}^{0} |f'(t)| dt
\le 
\varepsilon \int\limits_{x}^{0} |t^n| dt
\]
Comme $t \mapsto t^n$ est une fonction négative sur $\realIntervalCC{-\alpha}{0}$, on a peut alors changer les bornes de l'intégrale et on retombe alors sur la forme précédente.
\[
|f(x)-f(0)| 
\le 
\varepsilon \int\limits_{x}^{0} -t^n dt
\le 
\varepsilon \int\limits_{0}^{x} t^n dt
\le \epsilon x^{n+1}
\]
Ainsi, on a 
\[
\forall x \in \realIntervalCO{-\alpha}{\alpha}, 
|f(x)-f(0)| \le \dfrac{\varepsilon}{n+1} x^{n+1} 
\le 
\varepsilon x^{n+1}
\]
\end{proof}

\addtocounter{nbexercice}{100}

\section{Trigonométrie du cercle}

\subsection{Addition et multiple d'angle}

\begin{definition}[Sous espace propre]
Soient $E$ un $\mathbb{K}$ espace vectoriel de dimension quelconque, $u \in \mathscr{L}(E)$ et $\lambda \in \mathbb{K}$, alors on appelle \textbf{sous espace propre} associé à $\lambda$ l'ensemble 
\[
E_\lambda=\ker(u-\lambda Id_E)=\set{x \in E / u(x)= \lambda x }
\]
\end{definition}


\begin{exercice}
Soient $a, b, c \in \setRealNumbers$ tels que $\min\set{ab, bc, ca} > 1$. Montrer que 
\[
\sqrt[3]{(a^2+1)(b^2+1)(c^2+1)} \le \left( \dfrac{a + b + c}{2} \right)^2 + 1
\]
\end{exercice}

\begin{lemma}
\label{somme:suite:geometrique}
Soit $z \in \setComplexNumbers$ et $n \in \setNaturalNumbers$, alors 
\[
\displaystyle\sum\limits_{k=0}^{n} z^k =
\left\{
\begin{array}{l l}
\dfrac{1-z^{n+1}}{1-z} &\text{ si } z \ne 1\\
n + 1 &\text{ si } z = 1
\end{array}
\right.
\]
\end{lemma}

\begin{proof}
Soient $n \in \setNaturalNumbers$ et $z \in \setComplexNumbers$

\textbf{Cas 1 :} Supposons que $z \ne 1$, alors on peut appliquer le résultat précédent a $z$ et à $1$ : 
\[
1 - z^{n+1} = (1-z)\times\displaystyle\sum\limits_{k=0}^{n} z^k \underbrace{1^{n - k}}_{=1}
\]
Donc 
\[
\displaystyle\sum\limits_{k=0}^{n} z^k = \dfrac{1-z^{n+1}}{1-z}
\]

\textbf{Cas 2 :} Supposons que $z = 1$
\[
\displaystyle\sum\limits_{k=0}^{n} z^k = \displaystyle\sum\limits_{k=0}^{n} 1 = (n+1) \times 1 = n + 1
\]
\end{proof}

\begin{theorem}[Binôme de Newton]
\label{somme:binome:Newton}
Soient $x, y \in \setComplexNumbers$ et $n \in \setNaturalNumbers$,
\[
(a+b)^n = \displaystyle\sum\limits_{k=0}^{n} \displaystyle\binom{n}{k}a^k b^{n - k}
\]
\end{theorem}

\begin{proof}
On raisonne par récurrence,

\textbf{Initialisation :} $(a+b)^0 = 1$ et $\displaystyle\sum\limits_{k=0}^{0} \binom{0}{k}a^0 b^{- k} = 1 \times a^0 b^0 = 1$ donc $\mathcal{P}(0)$ est vérifiée.

\textbf{Hérédité :} On suppose que $\mathcal{P}(n) : (a+b)^n = \displaystyle\sum\limits_{k=0}^{n} \binom{n}{k}a^k b^{n - k}$ est vérifiée pour un certain rang $n$. \\
Montrons que $(a+b)^{n+1} = \displaystyle\sum\limits_{k=0}^{n+1} \binom{n+1}{k}a^k b^{n+1 - k}$ :\\
On a par hypothèse de récurrence :
\[
(a+b)^{n+1} = (a+b) (a+b)^{n} = (a+b) \displaystyle\sum\limits_{k=0}^{n} \binom{n}{k}a^k b^{n - k}
\]
on développe le produit. Par linéarité de la somme, on obtient
\[
(a+b)^{n+1} =  \displaystyle\sum\limits_{k=0}^{n} \binom{n}{k}a^{k+1} b^{n - k} + \displaystyle\sum\limits_{k=0}^{n} \binom{n}{k}a^{k} b^{n -1 - k}
\]
On ré-indice la première somme $k \leftarrow k+1$, donc
\[
(a+b)^{n+1} =  \displaystyle\sum\limits_{k=1}^{n+1}\binom{n}{k-1}a^{k} b^{n - 1 - k} + \displaystyle\sum\limits_{k=0}^{n} \binom{n}{k}a^{k} b^{n -1 - k}
\]
On sépare le premier et le dernier somme pour pouvoir regrouper tous les autres dans une même somme :
\[
(a+b)^{n+1} = \binom{n}{0}a^{0} b^{n +1} + \binom{n}{n}a^{n+1} b^{0} +\displaystyle\sum\limits_{k=0}^{n} \left(\binom{n}{k-1}+\binom{n}{k}\right)a^{k} b^{n +1 - k}
\]
Les coefficients binomiaux vérifient la relation de Pascal donc $\displaystyle\forall k \in \integerIntervalCC{1}{n}, \binom{n}{k-1}+\binom{n}{k}=\binom{n+1}{k}$\\
\[
(a+b)^{n+1} = \displaystyle\sum\limits_{k=0}^{n+1} \binom{n+1}{k}a^k b^{n+1 - k}
\]
\end{proof}

\begin{proof}
On a $\dfrac{d}{dx}\Bigl\langle\exp \circ \left(i\:  \textrm{Id}\right)\Bigr\rangle(x) = i\: e^{i\:x}$\\
On peut réécrire l'exponentielle complexe sous sa forme trigonométrique, ce qui nous donne par linéarité :
\[
\dfrac{d}{dx}\Bigl\langle\exp \circ \left(i\:  \textrm{Id}\right)\Bigr\rangle(x) 
= \dfrac{d}{dx}\Bigl\langle \cos +i\:\sin \Bigr\rangle(x) 
= \dfrac{d}{dx}\Bigl\langle \cos \Bigr\rangle(x) +i\:\dfrac{d}{dx}\Bigl\langle \sin \Bigr\rangle(x) 
\]
Donc comme $i^2 = 1$, 
\[
i\:e^{i\:x} = i \times (\cos(x)+i\:\sin(x))=-\:\sin(x)+i\:\cos(x)
\]

En associant partie réelle et partie imaginaire, on obtient alors :
\[
\dfrac{d}{dx}\Bigl\langle \cos \Bigr\rangle(x) =-\sin(x) 
\qquad\quad 
\dfrac{d}{dx}\Bigl\langle \sin \Bigr\rangle(x) =\cos(x)
\]
Pour la tangente, on applique les formules des quotients :
\[
\dfrac{d\tan}{dx}
=\dfrac{d}{dx}\Bigl\langle\dfrac{\sin}{\cos}\Bigr\rangle
=\dfrac{\dfrac{d}{dx}\Bigl\langle\sin\Bigr\rangle\cos - \sin\dfrac{d}{dx}\Bigl\langle\cos\Bigr\rangle}{\cos^2}
=\dfrac{\cos\cos + \sin\sin}{\cos^2}
=\dfrac{\cos^2 + \sin^2}{\cos^2}
=\dfrac{1}{\cos^2}
\]
De plus, on a par distributivité
\[
\dfrac{d}{dx}\Bigl\langle\tan\Bigr\rangle = \dfrac{\cos^2 + \sin^2}{\cos^2} = \dfrac{\cos^2}{\cos^2} + \dfrac{\sin^2}{\cos^2} = 1 + \tan^2
\]
\end{proof}

\begin{exercice}[Calcul de sommes polynomiales]
Soit $n \in \setNaturalNumbers$, calculer :
\[
\displaystyle\sum\limits_{k=1}^{n} k
\qquad\qquad\qquad
\displaystyle\sum\limits_{k=1}^{n} k^2
\qquad\qquad\qquad
\displaystyle\sum\limits_{k=1}^{n} k^3
\]
\end{exercice}
\begin{proof}
\textbf{Méthode 1 : Par récurrence}\\
Démontrons que 
\[
\forall n \in \setNaturalNumbers, \mathcal{P}(n) : \displaystyle\sum\limits_{k=0}^{n} k = \dfrac{n(n+1)}{2} 
\] 
Le résultat ne change pas si on ajoute le terme pour $k=0$ car il est nul. 

\textbf{Initialisation :} Pour $n = 0$, on a d'une part, $\displaystyle\sum\limits_{k=0}^{0} k = 0$ et d'autre part, on a
$\dfrac{0(0+1)}{2} = 0$. \\
Donc $\displaystyle\sum\limits_{k=0}^{n} k = \dfrac{0(0+1)}{2}$ i.e., $\mathcal{P}(0)$ est vraie.

\textbf{Hérédité :} Supposons que $\mathcal{P}(n)$ est vérifiée pour un certain rang $n \in \setNaturalNumbers$ alors, démontrons $\mathcal{P}(n+1)$ :\\
Remarquons que 
\[\displaystyle\sum\limits_{k=0}^{n+1} k= 0 + 1 + 2 + \cdots + n + (n+1)= \Bigl(1+2+ \cdots + n\Bigr) + \Bigl(n+1\Bigr) = \left(\displaystyle\sum\limits_{k=0}^{n} k\right) + (n+1)
\]
Or, d'après l'hypothèse de récurrence, on a $\mathcal{P}(n) : \displaystyle\sum\limits_{k=0}^{n} k =\dfrac{n(n+1)}{2}$.\\
Ainsi, 
\[
\displaystyle\sum\limits_{k=0}^{n+1} k = (n+1) + \dfrac{n(n+1)}{2} = \dfrac{(n+1)}{2}\times (2 + n) = \dfrac{(n+1)((n+1)+1)}{2}
\]
Ce qui est bien la formule qu'on a supposée, appliquée en $n+1$. Donc si  $\mathcal{P}(n+1)$ est vérifié, alors $\mathcal{P}(n+1)$ est vérifiée aussi.

\textbf{Conclusion :} $\lrboxed{\forall n \in \setNaturalNumbers,\displaystyle\sum\limits_{k=0}^{n} k = \dfrac{n(n+1)}{2}}$.

\textbf{Méthode 2 : Par l'utilisation d'un polynôme de télescopage}\\
Cette méthode permet de trouver les formules à démontrer contrairement à la récurrence. On va chercher un polynôme $P$ tel que $P(X+1) - P(X) = X$ en raisonnant par analyse-synthèse. On impose une condition qui va simplifier notre problème en ne considérant que les polynômes de degré $2$.

\textbf{Étape 1 : Analyse : La recherche du polynôme}\\
Supposons qu'il existe un tel polynôme $P$ alors comme $P$ est de degré $2$, il existe $a, b, c \in \setRealNumbers$ tels que $P = a\:X^2 + b\:X + c$.
On calcule explicitement $P(X+1) - P(X)$ : 
\[
P(X+1) - P(X) = a\:(X+1)^2 + b\:(X+1) + c - (a\:X^2 + b\:X + c) = 2a\:X + (a + b)
\]
Comme $P$ vérifie $P(X+1)-P(X)=X$, alors 
\[
P(X+1) - P(X) = 2a\:X + (1 + b)
\]
En identifiant les coefficients, on doit nécessairement avoir $a = \dfrac{1}{2}$ et $b= -a = -\dfrac{1}{2}$. \\
Ainsi, si $P$ est solution du problème, alors 
\[
P \in \set{\dfrac{1}{2} X^2 - \dfrac{1}{2}X + c\quad,\: c \in \setRealNumbers}
\] 

\textbf{Synthèse :} Soit $c \in \setRealNumbers$. Posons $a =\dfrac{1}{2}$ et $b =- \dfrac{1}{2}$ et définissons 
\[
P = a\:X^2  + b\:X + c = \dfrac{1}{2} X^2 - \dfrac{1}{2}X + c
\]
$P$ vérifie $P(X+1) - P(X) = 2\:X$ d'après ce que nous avons vu dans analyse. $c$ n'intervient pas dans l'égalité, donc prenons arbitrairement $c = 0$ car cela permet d'avoir $P(0)=c=0$ et de simplifier les calculs par la suite. 

\textbf{Étape 2 :} Soient $n \in \setNaturalNumbers$ et $k \in \integerIntervalCC{0}{n}$, alors 
\[
P(k+1) - P(k) = 2\:k
\]
Ainsi, si on somme pour toutes les valeurs de $k$ de $0$ jusqu'à $n$ cette égalité on obtient :
\[
\displaystyle\sum\limits_{k=0}^{n} \Bigl(P(k+1) - P(k)\Bigr) 
\:= \:
\displaystyle\sum\limits_{k=0}^{n} \: k 
\:\:= \:\:
2 \displaystyle\sum\limits_{k=0}^{n} k 
\]

\textbf{Étape 3 :} Or d'autre part, par linéarité de la somme
\[
\displaystyle\sum\limits_{k=0}^{n} \Bigl(P(k+1) - P(k)\Bigr) 
= \displaystyle\sum\limits_{k=0}^{n} P(k+1) - \displaystyle\sum\limits_{k=0}^{n} P(k)
\]
En posant $i = k+1$ dans la première somme, on obtient :
\[
=\displaystyle\sum\limits_{i=1}^{n+1}{P(i)} - \displaystyle\sum\limits_{k=0}^{n}{P(k)}
=\Bigl[P(1) + P(2) + ... + P(n) + P(n+1)\Bigr] - \Bigl[ P(0) + P(1) + ... + P(n)\Bigr]
\]
On remarque alors qu'il existe un télescopage et comme $P(0) = c = 0$, on obtient:
\[
\displaystyle\sum\limits_{k=0}^{n} \Bigl(P(k+1) - P(k)\Bigr) 
=P(n+1) - P(0) = P(n+1)
\]
Donc
\[
\displaystyle\sum\limits_{k=0}^{n} k = P(n+1) = \dfrac{1}{2}(n+1)^2 - \dfrac{1}{2}(n+1) = \dfrac{1}{2}(n+1)(n+1 - 1) = \dfrac{n(n+1)}{2}
\]
D'où 
\[
\displaystyle\sum\limits_{k=0}^{n}{k} = \dfrac{n(n+1)}{2}
\]
\end{proof}


\pagestyle{exercice}
\begin{exercice}
Résoudre suivant les valeurs de $a\in\setRealNumbers$ le système
\[
\left\{\begin{array}{l}
x+y+z+t=0\\
3x+ay+2z+t=a\\
2x+2y+z+3t=0\\
x+5y-z+4t=5\\
\end{array}\right.
\]
\end{exercice}

Soit $a \in \setRealNumbers$, soit $\begin{pmatrix}x\\y\\z\\t\end{pmatrix} \in \setRealNumbers^4$, alors \\
$\left\{\begin{array}{l}
x+y+z+t=0\\
3x+ay+2z+t=a\\
2x+2y+z+3t=0\\
x+5y-z+4t=5\\
\end{array}\right.$ ssi



\begin{exercice}
Déterminer une condition nécessaire et suffisante sur $\lambda\in\setRealNumbers$ pour que le système linéaire :
\[\left\{\begin{array}{l}
\lambda x+2y=0\\
x-\lambda y+z=0\\
2y-\lambda z=0\\
\end{array}\right.
\]
n'admette pas que la solution nulle
\end{exercice}

\begin{exercice}
Soient $a, b \in\setRealNumbers$, discuter la nature géométrique de l'ensemble des solutions de 
\[
\left\{\begin{array}{l}
ax+by+z=1\\
x+aby+z=b\\
x+by+az=1
\end{array}\right.
\]
\end{exercice}

\begin{exercice}
Soient $a,b,c,d\in\setRealNumbers$. Discuter la nature géométrique de l'ensemble des solutions de 
\[
\left\{\begin{array}{l}
x+y+z=1\\
ax+by+cz=d\\
a^2 x+b^2 y+c^2 z=d^2
\end{array}\right.
\]
\end{exercice}


\begin{exercice}
Calculer $\displaystyle\sum\limits_{k=1}^{n} k \ln\left(1+\dfrac{1}{k}\right)$ et $\displaystyle\sum\limits_{k=1}^{n}k \, k!$ en faisant apparaître des sommes télescopiques 
\end{exercice}

\begin{solution}
Soit $n \in \setNaturalNumbers$, on note $A_n = \displaystyle\sum\limits_{k=1}^{n} k \ln\left(1+\dfrac{1}{k}\right)$. D'après les propriétés de la fonction logarithme, on a 
\[
\forall k \in \integerIntervalCC{1}{n}, 
\ln\left(1+\dfrac{1}{k}\right)
= \ln\left(\dfrac{k+ 1}{k}\right) 
= \ln(k+1) - \ln(k) 
\]
Ainsi, 
\[
\forall k \in \integerIntervalCC{1}{n}, 
k  \ln\left(1+\dfrac{1}{k}\right)
= k  \ln(k+1) - k \ln(k) 
= (k + 1)  \ln(k+1) - k \ln(k) - \ln(k)
\]
La dernière factorisation permet de faire apparaître un télescopage. Notons
\[
\forall k \in \integerIntervalCC{1}{n}, u_k :=  k \ln(k)
\]
Alors, 
\[
A_n 
= \displaystyle\sum\limits_{k=1}^{n} u_{k+1} - u_{k} - \ln(k) =  \displaystyle\sum\limits_{k=1}^{n} (u_{k+1} - u_{k}) - \displaystyle\sum\limits_{k=1}^{n}\ln{k}
\]
Par télescopage, on obtient 
\[
A_n = (n+1)\ln(n+1) - 1 \times 0 - \ln\left(\displaystyle\prod\limits_{k=1}^{n} k\right) = \ln\left(\dfrac{(n+1)^{n+1}}{n!}\right)
\]
\end{solution}

\begin{exercice}
Calculer 
\[
\displaystyle\prod\limits_{p=2}^{n} \dfrac{p^3-1}{p^3+1}
\]
\end{exercice}

\begin{solution}
Soit $n \ge 2$ et $p \in \integerIntervalCC{2}{n}$, on remarque que l'on peut factoriser la fraction rationnelle de la manière suivante :
\[
\dfrac{p^3-1}{p^3+1} 
= \dfrac{p-1}{p+1} \times \dfrac{p^2 + p + 1}{p^2 - p + 1} 
= \dfrac{p-1}{p+1} \times \dfrac{(p+1)^2 - (p + 1) + 1}{p^2 - p + 1}
\]
Par associativité du produit, on a :
\[
\displaystyle\prod\limits_{p=2}^{n} \dfrac{p^3-1}{p^3+1} 
=
\displaystyle\prod\limits_{p=2}^{n} \dfrac{p-1}{p+1} \times \dfrac{(p+1)^2 - (p + 1) + 1}{p^2 - p + 1}
= \left(\displaystyle\prod\limits_{p=2}^{n} \dfrac{p-1}{p+1}\right) 
\times 
\left(\displaystyle\prod\limits_{p=2}^{n} \dfrac{(p+1)^2 - (p + 1) + 1}{p^2 - p + 1}\right)
\]
Notons $A_n = \displaystyle\prod\limits_{p=2}^{n} \dfrac{p-1}{p+1}$ la première partie du produit et $B_n=\dfrac{(p+1)^2 - (p + 1) + 1}{p^2 - p + 1}$ la seconde.
On a alors par linéarité, puis par définition de la factorielle :
\[
\lboxed{A_n} 
= \displaystyle\prod\limits_{p=2}^{n}\dfrac{p-1}{p+1} 
= \dfrac{\displaystyle\prod\limits_{p=2}^{n}(p-1)}{\displaystyle\prod\limits_{p=2}^{n}(p+1)} 
= \dfrac{(n-1)!}{\dfrac{(n+1)!}{2!}} 
\rboxed{= \dfrac{2}{n(n+1)}}
\]
D'autre part, le second produit est un produit télescopique évident, donc
\[
\lboxed{B_n}
= \displaystyle\prod\limits_{p=2}^{n}\dfrac{(p+1)^2 - (p + 1) + 1}{p^2 - p + 1} 
= \dfrac{(n+1)^2 - (n + 1) + 1}{(2)^2 - (2) + 1} 
\rboxed{= \dfrac{n^2 + n + 1}{3}}
\]
Ainsi, on obtient 
\[
\lboxed{\displaystyle\prod\limits_{p=2}^{n}\dfrac{p^3-1}{p^3+1}} 
= A_n \times B_n 
\rboxed{= \dfrac{2}{3}\times \dfrac{n^2+n+1}{n(n+1)}}
\]
\end{solution}

\begin{exercice}
Soit $S_n=\displaystyle\sum\limits_{k=1}^{n} k$, calculer rapidement $\displaystyle\sum\limits_{k=1}^{2n-1} S_k$
\end{exercice}


\begin{exercice}
Soient $n,m\in\setNaturalNumbers\left/n \le m\right.$. Que représente sur le triangle de Pascal la somme $\displaystyle\sum\limits_{k=n}^{m} \binom{k}{n}$ ?\\
Conjecturez la valeur de cette somme puis prouver le résultat.
\end{exercice}

\begin{solution}
Cette somme correspond à la somme d'une colonne du triangle de Pascal. D'après la formule de la relation de Pascal, on a 
\[
\forall k \in \integerIntervalCC{n+1}{m},\binom{k}{n} + \binom{k}{n+1} = \binom{k+1}{n+1} 
\]
Donc on a
\[
\forall k \in \integerIntervalCC{n+1}{m},\binom{k}{n} = \binom{k+1}{n+1} - \binom{k}{n+1}
\]
En sommant les égalités pour $k=n+1$ jusqu'à $m$ (sans le premier terme), on a alors
\[
\displaystyle\sum\limits_{k=n+1}^{m} \binom{k}{n} 
= \displaystyle\sum\limits_{k=n+1}^{m} \left(\binom{k+1}{n+1} - \binom{k}{n+1}\right)
\]
On remarque qu'il y a un télescopage qui apparaît. On obtient alors
\[
\displaystyle\sum\limits_{k=n+1}^{m+1} \binom{k}{n+1}
- \displaystyle\sum\limits_{k=n}^{m} \binom{k+1}{n}
= \binom{m+1}{n+1} - \underbrace{\binom{n+1}{n+1}}_{=1}
= \binom{m+1}{n+1} - 1
\]
Donc 
\[
\lboxed{\displaystyle\sum\limits_{k=n+1}^{m} \binom{k}{n}}
= \binom{n}{n} + \displaystyle\sum\limits_{k=n+1}^{m} \binom{k}{n} 
= \underbrace{\binom{n}{n}}_{=1} + \left(\binom{m+1}{n+1} - 1\right)
\rboxed{= \displaystyle\binom{m+1}{n+1}}
\]
\end{solution}

\begin{exercice}
Calculer 
\[
\displaystyle\sum\limits_{k=0}^{n} \dfrac{1}{k+1}\binom{n}{k}
\]
\end{exercice}

\begin{solution}
Soit $n \in \setNaturalNumbers$, on a d'après la relation de la diagonale du binôme :
\[\forall k \in \integerIntervalCC{0}{n}, \binom{n+1}{k+1} = \dfrac{n+1}{k+1}\binom{n}{k}
\]
On en déduit que
\[
\forall k \in \integerIntervalCC{0}{n}, \dfrac{1}{k+1}\binom{n}{k} = \dfrac{1}{n+1}\binom{n+1}{k+1}
\]
Ainsi, en remplaçant par cette nouvelle expression le terme général de la somme, on obtient :
\[
\lboxed{\displaystyle\sum\limits_{k=0}^{n} \dfrac{1}{k+1}\binom{n}{k}}
=  \displaystyle\sum\limits_{k=0}^{n} \dfrac{1}{n+1}\binom{n}{k} 
= \dfrac{1}{n+1} \displaystyle\sum\limits_{k=0}^{n} \binom{n}{k}
\rboxed{= \dfrac{2^n}{n+1}}
\]
\end{solution}

\begin{exercice}
Calculer 
\[
\displaystyle\sum\limits_{k=1}^{n} \dfrac{k}{2^{2k}}\binom{n}{k}
\]
\end{exercice}

\begin{solution}
Soit $n \in \setNaturalNumbers$,
\[
\displaystyle\sum\limits_{k=1}^{n} \dfrac{k}{2^{2k}}\binom{n}{k} = \displaystyle\sum\limits_{k=1}^{n}k \binom{n}{k} \left(\dfrac{1}{4}\right)^k
\]
Or on sait d'après la relation de la diagonale que $\forall k \in \integerIntervalCC{1}{n}, \displaystyle\binom{n}{k} = \dfrac{n}{k}\binom{n-1}{k-1}$, en réinjectant cette égalité dans la somme, on obtient :
\[
\displaystyle\sum\limits_{k=1}^{n}k \binom{n}{k} \left(\dfrac{1}{4}\right)^k
=\displaystyle\sum\limits_{k=1}^{n}\cancel{k} \dfrac{n}{\cancel{k}}\binom{n-1}{k-1} \left(\dfrac{1}{4}\right)^k
=n\displaystyle\sum\limits_{k=1}^{n} \binom{n-1}{k-1} \left(\dfrac{1}{4}\right)^k
\]
En effectuant un changement d'indice ($k \leftarrow k+1$) et en ajustant la puissance sur $\dfrac{1}{4}$, on fait apparaître le binôme de Newton :
\[
n\displaystyle\sum\limits_{k=1}^{n} \binom{n-1}{k-1} \left(\dfrac{1}{4}\right)^k 
= n\displaystyle\sum\limits_{k=0}^{n-1} \binom{n-1}{k} \left(\dfrac{1}{4}\right)^{k+1} =  n\left(\dfrac{1}{4}\right)\displaystyle\sum\limits_{k=0}^{n-1} \binom{n-1}{k} \left(\dfrac{1}{4}\right)^{k}
\]
En appliquant la formule du binôme à $1$ et $\dfrac{1}{4}$, on en déduit que :
\[
\lrboxed{\displaystyle\sum\limits_{k=1}^{n} \dfrac{k}{2^{2k}}\binom{n}{k} =\dfrac{n}{4}\left(\dfrac{5}{4}\right)^{n-1}}
\]
\end{solution}

\begin{exercice}
Calculer 
\[
\displaystyle\sum\limits_{k=0}^{n} 2^k \binom{2n}{2k}
\]
\end{exercice}

\begin{solution}
Soit $n \in \setNaturalNumbers$, posons $A_n = \displaystyle\sum\limits_{k=0}^{n} \sqrt{2}^{2k} \binom{2n}{2k}$ et $B_n = \displaystyle\sum\limits_{k=0}^{n-1} \sqrt{2}^{2k+1}  \binom{2n}{2k+1}$.\\
Alors $A_n = \displaystyle\sum\limits_{k=0}^{n} 2^k \binom{2n}{2k}$ et si on note pour tout $ p \in \integerIntervalCC{0}{2n}, u_p = \sqrt{2}^{p} \displaystyle\binom{2n}{p}$, $A_n$ se réécrit $\displaystyle\sum\limits_{k=0}^{n} u_{2k}$. De la même manière, $B_n$ se réécrit $\displaystyle\sum\limits_{k=0}^{n-1} u_{2k+1}$.\\
D'après la formule du binôme de Newton, on a
\[
\lboxed{A_n + B_n} 
= \displaystyle\sum\limits_{k=0}^{n} u_{k} 
= \displaystyle\sum\limits_{k=0}^{2n}\binom{2n}{k} \sqrt{2}^k 
\rboxed{= (1+\sqrt{2})^{2n}} 
\]
D'autre part,  
\[
\lboxed{A_n - B_n}
= \displaystyle\sum\limits_{k=0}^{2n} (-1)^k u_{k} 
= \displaystyle\sum\limits_{k=0}^{2n} \binom{2n}{k}  (-\sqrt{2})^k 
\rboxed{= (1-\sqrt{2})^{2n}}
\]
En fin de compte,  
\[
\lboxed{\displaystyle\sum\limits_{k=0}^{n} 2^k \binom{2n}{2k}}
= A_n 
= \dfrac{(A_n+B_n) + (A_n - B_n)}{2} 
\rboxed{= \dfrac{(1+\sqrt{2})^{2n}+(1-\sqrt{2})^{2n}}{2}}
\]

On en déduit que \hfil$\displaystyle\sum\limits_{k=0}^{\floor{n}} 2^k \binom{n}{2k} = \dfrac{(1+\sqrt{2})^{n}+(1-\sqrt{2})^{n}}{2}$
\end{solution}


\begin{exercice}
Montrer que $\forall x,y\in\setRealNumbers^+,\sqrt{x+y}\le\sqrt{x}+\sqrt{y}$
\end{exercice}

\begin{solution}
Soient $x, y \in \setRealNumbers^+$, alors $x+y \ge 0$ donc $\sqrt{x}$, $\sqrt{y}$ et $\sqrt{x+y}$ sont bien définis. De plus, comme une racine carré est toujours positive, on a donc l'équivalence en passant au carré
\[
0\le\sqrt{x+y}\le\sqrt{x}+\sqrt{y}
\quad\Leftrightarrow\quad
x + y \le x + 2 \sqrt{x}\sqrt{y} + y 
\quad\Leftrightarrow\quad
0 \le \sqrt{xy}
\] 
Cette dernière condition étant vérifiée pour tout $x$ et $y$ positifs, on a bien l'égalité recherché.
\end{solution}

\begin{exercice}
Montrer que $\forall x,y\in\setRealNumbers, \forall\lambda>0, 2xy\le\dfrac{x^2}{\lambda}+\lambda y^2$
\end{exercice}

\begin{solution}
Soient $x, y \in \setRealNumbers$ et $\lambda \in \setRealNumbers^{+*}$, on a \[
\left(\dfrac{x}{\sqrt{\lambda}} - \sqrt{\lambda} y\right)^2 = \dfrac{x^2}{\lambda} - \underbrace{2 \times \dfrac{x}{\sqrt{\lambda}}\times \sqrt{\lambda} y}_{=2xy} +\lambda y^2
\]
Or on sait qu'un carré est toujours positif. Donc on a
\[
\dfrac{x^2}{\lambda} - 2xy +\lambda y^2 \ge 0
\]
C'est-à-dire que
\[
\dfrac{x^2}{\lambda} +\lambda y^2 \ge 2xy
\]
\end{solution}

\begin{exercice}
Résoudre l'inéquation $\left|\dfrac{x-2}{x+1}\right|<2$
\end{exercice}

\begin{solution}
Soit $x \in \setRealNumbers \setminus \set{-1}$,
On va distinguer les cas en fonction du signe de $\dfrac{x-2}{x+1}$.
\begin{itemize}
\item Supposons que $x-2 \ge 0$.
\begin{itemize}
\item Supposons que $x+1 \ge 0$.\\
Alors $x \ge 2$ et $x \ge -1$ donc $x \ge 2$.\\
De plus, $\dfrac{x-2}{x+1} \ge 0$ donc il faut résoudre l'équation $\dfrac{x-2}{x+1} \le 2$.\\
Cela revient à voir si $x-2 > 2(x+1)$ ie $-2-2 < x$, ce qui est toujours vrai comme $x \ge 2$. Donc l'ensemble $\realIntervalCO{2}{+\infty}$ est solution.
\item Supposons que $x+1 < 0$.\\
Alors $x \ge 2$ et $x < -1$, ce n'est pas possible.
\end{itemize}
\item Supposons que $x-2 < 0$.
\begin{itemize}
\item Supposons que $x+1 \ge 0$.\\
On a alors $x < 2$ et $x \ge -1$, sur cet intervalle, $\dfrac{x-2}{x+1} \le 0$.
Donc on doit résoudre $\dfrac{x-2}{x+1} > -2$. \\
C'est équivalent à $(x-2) > -2x - 2$ ($x+1$ est positif donc la multiplication conserve l'ordre), i.e. $0 >x$. Donc l'ensemble $\realIntervalOO{0}{2}$  est solution.
\item Supposons que $x+1 < 0$.
Alors $x < -1$. Dans ce cas, il faut résoudre l'équation $\dfrac{x-2}{x+1} < 2$.\\
Cela équivaut à dire que $x - 2 > 2x + 2$ ie $x < -4$. Donc l'ensemble $\realIntervalOO{-\infty}{-4}$ est solution
\end{itemize}
\end{itemize}
 
Finalement, en regroupant les intervalles solutions, on obtient $S = \realIntervalOO{-\infty}{-4} \cup \realIntervalCO{0}{+\infty}$.
\end{solution}

\begin{exercice}
Montrer que $\forall x \in \realIntervalCC{0}{1}, x-\dfrac{1}{6}x^3 \le \sin(x) \le x$
\end{exercice}
Posons $f(x) = \sin(x) - x$. \\
Alors $f'(x) = \cos(x) - 1 \le 0$
\begin{exercice}
Étudier la fonction $\function[f]{x}{2\:x + \ln(x+1)-\dfrac{1}{x}}$
\end{exercice}
\begin{exercice}
Étudier la fonction $\function[f]{x}{\dfrac{\ln(x)}{x}}$
\end{exercice}
\begin{exercice}
Étudier la fonction $\function[f]{x}{x e^{-x^2}}$
\end{exercice}
\begin{exercice}
Étudier la fonction $\function[f]{x}{x^2 e^x}$
\end{exercice}
\begin{exercice}
Étudier les variations de $\function[f]{x\in\setRealNumbers}{\dfrac{2x^3}{1+x^2}}$, ses asymptotes et la position de la courbe de $f$ par rapport à celles-ci.\\
En déduire un graphe précis de $f$.\\
Étudier le signe de $f(x)-x$ et tracer sur le graphe précédent la première bissectrice.
\end{exercice}
\begin{exercice}
Soit $n\in\setNaturalNumbers^*$. On pose $\function[f]{x>0}{x^{n-1} \ln(x)}$. Calculer $f^{(n)}(x)$.
\end{exercice}

\begin{solution}
On sait que :
\[
\forall k \in \setNaturalNumbers^*, \dfrac{d^k}{dx^k}\left\langle\ln\right\rangle(x) = \dfrac{(-1)^{k-1}(k-1)!}{x^k}
\]
Et que 
\[
\forall k \in \setNaturalNumbers^*, \dfrac{d^k}{dx^k}\left\langle Id^{n-1} \right\rangle(x) = \dfrac{(n-1)!}{(n-1-k)!} x^{n-1-k}
\]
D'après la formule du binôme de Leibniz, on a donc
\[
\forall x \in \setRealNumbers^{+*}, 
f^{(n)}(x)
= \displaystyle\sum\limits_{k=0}^{n} \binom{n}{k} (Id^{n-1})^{(n)}(x) \ln^{(n-k)}(x)
\]
Donc 
\[
\forall x \in \setRealNumbers^{+*}, 
f^{(n)}(x)
= \left(\displaystyle\sum\limits_{k=0}^{n-1} \binom{n}{k} \dfrac{(n-1)!}{(n-1-k)!} x^{(n-1)-k} \dfrac{(-1)^{n-k-1}(n-k-1)!}{x^{n-k}}\right) + 0 \times \ln(x)
\]
On remarque que $x^{n-k}$ et $x^{n-1-k}$ se simplifient en $x^{-1}$ et que les facteurs $(n-1-k)!$ se s'annulent :
\[
\displaystyle\sum\limits_{k=0}^{n-1} \binom{n}{k} (n-1)! \: x^{-1} (-1)^{n-k-1}
= \dfrac{(n-1)!}{x} \displaystyle\sum\limits_{k=0}^{n-1} \binom{n}{k} (-1)^{(n-1)-k}
\]
D'après la formule du binôme de Newton, on sait que 
\[
\displaystyle\sum\limits_{k=0}^{n} \binom{n}{k} (-1)^{n-k} = (1-1)^n \underbrace{= 0}_{\text{car $n > 0$}}
\]
Donc en sortant le dernier terme et en rajoutant $-1$ en puissance :
\[
(-1)\left(\displaystyle\sum\limits_{k=0}^{n-1} \binom{n}{k} (-1)^{n-1-k}\right) + \binom{n}{n} = 0
\]
On a alors 
\[
\displaystyle\sum\limits_{k=0}^{n-1} \binom{n}{k} (-1)^{n-1-k} = \binom{n}{n} = 1
\]
D'où finalement, 
\[
\lboxed{\forall x \in \setRealNumbers^{+*}, 
f^{(n)}(x)}
= \dfrac{(n-1)!}{x} \times 1 
\rboxed{= \dfrac{(n-1)!}{x}}
\] 
\end{solution}

\begin{exercice}
Soient $a,b\in\setRealNumbers$, posons $\function[f]{x\in\setRealNumbers}{(x-a)^n (x-b)^n}$. Calculer $f^{(n)}(x)$.\\
En déduire $\displaystyle\sum\limits_{k=0}^{n} \binom{n}{k}^2$
\end{exercice}

\begin{solution}
Soient $a,b\in\setRealNumbers$ et $n \in \setIntegers$, alors on note 
\[
g_a = x\longmapsto (x-a)^n \quad\text{et}\quad g_b = x \longmapsto (x-b)^n
\]
D'après la formule de Leibniz, comme $f = g_a g_b$ :
\[
\forall x \in \setRealNumbers,f^{(n)}(x) = \sum\limits_{k=0}^{n} \binom{n}{k} g_a^{(k)} g_b^{(n-k)}
\]
Or 
\[
\forall k \in \integerIntervalCC{0}{n}, g_a^{(k)} = \dfrac{n!}{(n-k)!} (x-a)^{n-k}
\]
Donc 
\[
\forall x \in \setRealNumbers,f^{(n)}(x) 
= \sum\limits_{k=0}^{n} \binom{n}{k}  \dfrac{n!}{(n-k)!} (x-a)^{n-k}  \dfrac{n!}{k!} (x-b)^{k}
= n! \sum\limits_{k=0}^{n} \binom{n}{k}^2  (x-a)^{n-k} (x-b)^{k}
\]
Si $a = b$, on a alors
\[
\forall x \in \setRealNumbers,f^{(n)}(x) 
= n! \sum\limits_{k=0}^{n} \binom{n}{k}^2 (x-a)^{n}
=n!(x-a)^{n}\sum\limits_{k=0}^{n} \binom{n}{k}^2
\]
D'autre part, $f = x \longmapsto (x-a)^n (x-a)^n = (x-a)^{2n}$, donc 
\[
\forall x \in \setRealNumbers,f^{(n)}(x) 
=  \dfrac{(2n)!}{(2n-n)!} (x-a)^{2n-n} =   \dfrac{(2n)!}{n!} (x-a)^{n}
\]
Ainsi, par unicité de la dérivée, on a 
\[
\forall x \in \setRealNumbers,
n!(x-a)^{n}\sum\limits_{k=0}^{n} \binom{n}{k}^2 = \dfrac{(2n)!}{n!} (x-a)^{n}
\]
Donc en divisant à gauche et à droite par $n!(x-a)^{n}$

\[
\lboxed{\displaystyle\sum\limits_{k=0}^{n} \binom{n}{k}^2}
= \dfrac{(2n)!}{n!n!}
= \dfrac{(2n)!}{n!(2n-n)!} 
\rboxed{= \displaystyle\binom{2n}{n}}
\]
\end{solution}

\newpage
\pagestyle{default}

%% Preuve dérivée trigonométriques
\begin{proof}
\textbf{Méthode géométrique}

\textbf{Étape 1 :} Limite de $\sin(x)/x$ en 0\\
Avec un dessin, on se rend compte qu'on a l'inclusion des aires suivantes :
\[
\forall x \in \realIntervalCO{0}{\dfrac{\pi}{2}}, \dfrac{\sin(x)\cos(x)}{2} \le \dfrac{x}{2} \le \dfrac{\tan(x)}{1}
\]
La première inégalité nous donne 
\[
\forall x \in \realIntervalCO{0}{\dfrac{\pi}{2}}, \sin(x)\cos(x) \le x
\]
Donc 
\[
\dfrac{\sin(x)}{x} \le \dfrac{1}{\cos(x)}
\]
La seconde inégalité nous donne 
\[
\forall x \in \realIntervalCO{0}{\dfrac{\pi}{2}}, x \le \tan(x) = \dfrac{\sin(x)}{\cos(x)}
\]
Donc 
\[
\cos(x) \le \dfrac{\sin(x)}{x} 
\]
Ainsi, on a
\[
\forall x \in \realIntervalCO{0}{\dfrac{\pi}{2}}, \cos(x) \le \dfrac{\sin(x)}{x} \le \dfrac{1}{\cos(x)}
\]
Comme $\cos$ sont continues sur $\setRealNumbers$, on peut faire tendre $x$ vers 0, on obtient alors par passage à la limite :
\[
\lim_{x \mapsto 0} \cos(x) = 1 \quad\text{ et }\quad \dfrac{1}{\cos(0)} = 1
\]
Donc $\displaystyle\lim_{x \longmapsto 0^+} \dfrac{\sin(x)}{x} = 1$ d'après le théorème des gendarmes. Comme $\sin$ est continue sur $\setRealNumbers$, la limite à gauche est égale à la limite à droite. On a donc 
\[
\displaystyle\lim_{x \longmapsto 0} \dfrac{\sin(x)}{x} = 1
\]

\textbf{Étape 2 :} Calcul du taux d'accroissement de $\cos$\\
Soient $x, h \in \setRealNumbers$, alors d'après les formules trigonométriques :
\[
\dfrac{\cos(x+h)-\cos(x)}{h} = \dfrac{-2\sin\left(\dfrac{(x+h)+x}{2}\right)\sin\left(\dfrac{(x+h)-x}{2}\right)}{h}
\]
Comme $\dfrac{(x+h)-x}{2} = \dfrac{h}{2}$ et que $\dfrac{(x+h)+x}{2} = x +\dfrac{h}{2}$, on a :
\[
\dfrac{\cos(x+h)-\cos(x)}{h} = \dfrac{-2\sin\left(x + \dfrac{h}{2}\right)\sin\left(\dfrac{h}{2}\right)}{h} = -\dfrac{2\sin\left(\dfrac{h}{2}\right)}{h} \times \sin\left(x + \dfrac{h}{2}\right)
\]
Or on a montré que $\displaystyle\lim_{x \longmapsto 0} \dfrac{\sin(x)}{x} = 1$ donc en appliquant ça à $\dfrac{h}{2}$, on a :
\[
\displaystyle\lim_{h \longmapsto 0} \dfrac{2\sin\left(\dfrac{h}{2}\right)}{h} = 1
\]
Comme $\sin$ est une fonction continue, on a d'autre part
\[
\displaystyle\lim_{h \longmapsto 0} \sin\left(x + \dfrac{h}{2}\right) = \sin(x)
\]
Ainsi, 
\[
-\left(\displaystyle\lim_{h \longmapsto 0} \dfrac{2\sin\left(\dfrac{h}{2}\right)}{h} \right)\times \left(\displaystyle\lim_{k \longmapsto 0} \sin\left(x + \dfrac{k}{2}\right)\right) = - \sin(x)
\]
Donc 
\[
\displaystyle\lim_{h \longmapsto 0} \dfrac{\cos(x+h)-\cos(x)}{h} = - \sin(x)
\]
On a ainsi montré que $\dfrac{d}{dx}\left\langle \cos \right\rbrace = -\sin$. On sait de plus que $\sin = \cos\left(\cdot - \dfrac{\pi}{2}\right)$. Par composition des dérivées, on en déduit que 
\[
\forall x \in \setRealNumbers,
\dfrac{d}{dx}\left\langle \sin \right\rangle(x) = \dfrac{d}{dx}\left\langle \cos\left(\dot - \dfrac{\pi}{2}\right) \right\rangle(x) = - \sin\left(x - \dfrac{\pi}{2}\right)
\]
Et 
\[
\forall x \in \setRealNumbers, - \sin\left(x - \dfrac{\pi}{2}\right) = - \left(\sin(x)\underbrace{\cos\left(\dfrac{\pi}{2}\right)}_{=0}-\cos(x)\underbrace{\sin\left(\dfrac{\pi}{2}\right)}_{=1}\right) = \cos(x)
\]
On en déduit alors que $\dfrac{d}{dx}\left\langle \sin \right\rangle = \cos$
\end{proof}

\begin{exercice}
Déterminer les limites suivantes :\\
\[
\lim_{x \rightarrow +\infty} \dfrac{e^{2x}(\ln(x))^3}{x^4}
\qquad\quad
\lim_{x \rightarrow 0^+} x^2(\ln(x^3))^3
\qquad\quad
\lim_{x \rightarrow -\infty} x^2 e^x(\ln(-x))^3
\]
\end{exercice}
\begin{exercice}
Résoudre l'équation 
\[
2x\ln(x)+3(x-1) = 0
\]
\end{exercice}

$ x = 1$
\begin{figure}[H]\centering
\begin{tikzpicture}
\begin{axis}[
xlabel=$x$,
ylabel={$f(x)$},
xmin = 0,
xmax = 1.4,
samples=100,
width=12cm%,%height=3cm
]
\addplot[red,domain=0:1.4]{2 *x * ln(x)+3 * (x-1)};
\addlegendentry{$f(x) =2x\ln(x)+3(x-1)$}
\draw [dashed] (0,0) -- (\pgfkeysvalueof{/pgfplots/xmax},0);
\end{axis}
\end{tikzpicture}
\caption{Graphique de $2x\ln(x)+3(x-1)$ sur $\realIntervalCC{0}{1.4}$}
\end{figure}

\begin{exercice}
Résoudre dans $\setRealNumbers$ l'équation 
\[
3^x+4^x=5^x
\]
\end{exercice}
\begin{exercice}
Montrer que 
\[
\forall x \in \realIntervalOO{0}{1}, x^x (1-x)^{1-x} \ge \dfrac{1}{2}
\]
\end{exercice}
\begin{exercice}
Montrer que 
\[\forall x,y\in\setRealNumbers, \left/ 0 < x < y \right.,\dfrac{y-x}{\ln(x)-\ln(y)} < \dfrac{x+y}{2}
\] 
Indication: On utilisera $t=\dfrac{y}{x}$
\end{exercice}
\begin{exercice}
Rappeler la formule trigonométrique $\sin(a+b)$. Montrer que
\[
\arcsin\left(\dfrac{5}{13}\right)+\arcsin\left(\dfrac{3}{5}\right)= \arcsin\left(\dfrac{56}{65}\right)
\]
\end{exercice}
\begin{exercice}
Montrer que 
\[
\forall x \ge 0,\dfrac{x}{1+x^2} \le \arctan(x) \le \dfrac{\pi}{2}-\dfrac{x}{1+x^2}
\]
\end{exercice}
\begin{exercice}
Quel est le domaine de définition de 
$\function[f]{x}{\arctan\left(\dfrac{1+x}{1-x}\right)}$ ?
Calculer sa dérivée. Conclusion ?
\end{exercice}
\begin{exercice}
Quel est le domaine de définition de $f: x \longmapsto \dfrac{2x}{1+x^2}$? Calculer sa dérivée. Conclusion ?
\end{exercice}
\begin{exercice}
Résoudre dans $\setRealNumbers^2$ le système 
\[
\left\{\begin{array}{l}
\cosh(x) + \sinh(y) = 4 \\
\sinh(x) + \sinh(y) = 1 \\
\end{array}\right.
\]
\end{exercice}
\begin{exercice}
Soit $y\in\setRealNumbers$. Résoudre l'équation 
\[
\sinh(x)=y
\]
\end{exercice}

\begin{exercice}
Montrer que 
\[
\setUnimodularComplexNumbers\setminus\{-1\}=\set{\dfrac{1+ia}{1-ia},a\in\setRealNumbers}
\]
\end{exercice}
\begin{exercice}
Résoudre l'équation 
\[
z+\dfrac{1}{\overline{z}}=1
\]
\end{exercice}
\begin{exercice}
Soit $(a,b)\in\setComplexNumbers^2$. Montrer que 
\[
|a|+|b|\le|a+b|+|a-b|
\]
et étudier le cas d'égalité.
\end{exercice}
\begin{exercice}
Soit $\omega=e^{2i\pi/5}$ et $x= \omega+\omega^{-1}$. Calculer $x^2+x-1$.\\
Que peut-on en déduire ?
\end{exercice}
\begin{exercice}
Résoudre l'équation 
\[
z^4-z^3+z^2+2=0
\]
en remarquant que $j$ est solution.
\end{exercice}
\begin{exercice}
Déterminer les racines quatrièmes de $-7-24i$.
\end{exercice}
\begin{exercice}
Résoudre dans $\setComplexNumbers$ l'équation 
\[
\overline{z}^n=z
\]
\end{exercice}
\begin{exercice}
Résoudre l'équation 
\[
(z^2+1)^{n} = (z-i)^{2n}
\]
\end{exercice}
\begin{exercice}
Soient $ABC$ et $ADE$ deux triangles équilatéraux directs et $ABCD$ un 
parallélogramme. Montrer que $BFE$ est équilatéral direct.
\end{exercice}
\begin{exercice}
Soient $n\in\setNaturalNumbers^*$ et $\omega$ une racine n-ième de l'unité. Calculer 
\[
\sum\limits_{k=0}^{n-1} (k+1) \omega^k
\]
\end{exercice}
\begin{exercice}
Déterminer une CNS sur $(a,b)\in\setComplexNumbers^2$ pour que les deux racines de $z^2+az+b=0$ aient même module.
\end{exercice}

\begin{exercice}
Résoudre les équations 
\[
\left\{\begin{array}{l}
\cos(5x)+2\cos(3x)+3\cos(x) = 0\\
\cos(x)-\cos(2x) = \sin(3x)
\end{array}\right.
\]
\end{exercice}
\begin{exercice}
Résoudre l'équation 
\[
\cos(x)+\sin(x)=\tan\left(\dfrac{x}{2}\right)
\]
en posant $t=\tan\left(\dfrac{x}{2}\right)$
\end{exercice}
\begin{exercice}
Étudier la fonction 
\[
\function[f]{x}{\ln\left(\tan\left(\dfrac{x}{2}\right)\right)+\sin(x)}
\]
\end{exercice}
\begin{exercice}
Montrer que 
\[
\forall n \in\setNaturalNumbers, \forall t \in \setRealNumbers, |\sin(nt)|\le n |\sin(t)|
\]
\end{exercice}
\begin{exercice}
Étudier la fonction 
\[
\function[f]{x}{\ln(\cosh(x))}
\]
\end{exercice}
\begin{exercice}
Calculer 
\[
\sum\limits_{k=0}^{n-1} 3^k \sin^3 \left(\dfrac{\theta}{3^{k+1}}\right)
\]
\end{exercice}
\begin{exercice}
Calculer 
\[
\sum\limits_{k=0}^{n} k\binom{n}{k}\cos(k\theta)
\]
\end{exercice}
\begin{exercice}
Calculer les primitives suivantes en précisant à chaque fois le ou les intervalles où ceci est légitime :
\[
\int e^{-\sqrt{x}} dx
\qquad 
\int \cos(\ln(x)) dx
\qquad 
\int \ln(1+x^2) dx
\]
\[
\int \arctan(x) dx 
\qquad
\int x (\sin(x))^2 dx
\qquad 
\int \dfrac{x^2}{(x-1)^2(x^2+4)} dx
\] 
\[
\int \cos(2x)\cos^2(x)
\qquad
\int \dfrac{1}{\sin(t)+ \tan(t)} dx \text{ sur } \realIntervalCC{0}{\dfrac{\pi}{2}} \text{ en posant } u=\cos(x)
\]
\end{exercice}
\begin{exercice}
Posons 
\[
\forall n \in \setNaturalNumbers, I_n := \int \dfrac{1}{(1+x)^n} dx
\]
À l'aide d'une intégration par parties, obtenir une relation de récurrence entre $I_n$ et $I_{n+1}$. \\
En déduire $I_1, I_2$ et $I_3$ 
\end{exercice}
\begin{exercice}
Posons 
\[
\forall n \in \setNaturalNumbers, I_n = \int (1-x^2)^n dx
\]
\begin{enumerate}
\item Établir la relation de récurrence entre $I_n$ et $I_{n+1}$
\item Calculer $I_n$
\item En déduire $\displaystyle\sum\limits_{k=0}^{n} \dfrac{(-1)^k}{2k+1}\binom{n}{k}$
\end{enumerate}
\end{exercice}


\begin{exercice}
Résoudre 
\[
y'+2y=2\sinh(2x)
\]
\end{exercice}
\begin{exercice}
Résoudre 
\[
\sin(x)y'=\cos(x)y+2x\sin^2(x) \text{ sur }\realIntervalOO{0}{\pi}
\]
\end{exercice}
\begin{exercice}
Résoudre 
\[
y''+y'+y = e^{-x/2}\cos\left(\dfrac{\sqrt{3}}{2}x\right)
\]  
\end{exercice}
\begin{exercice}
Résoudre 
\[
y''-2y'+2y = e^x \sin^2(x)
\] 
avec $y(0)=0$ et $y'(0)=1$
\end{exercice}
\begin{exercice}
Résoudre le système différentiel 
\[
\left\{\begin{array}{l}
x'(t)=x(t)+y(t)\\
y'(t)=3 x(t)-y(t)
\end{array}\right.
\]
avec $x(0)=2$ et $y(0)=-2$.
\end{exercice}
\begin{exercice}
Déterminer les $f\in\setContinuousFunctions{0}{\setRealNumbers}{\setRealNumbers}$ telles que 
\[
\forall x\in\setRealNumbers, f(x)+\int_{0}^{x} t f(t) dt = 1
\]
\end{exercice}
\begin{exercice}
Résoudre 
\[
xy''-y'-x^3 y(x) = 0 
\]
sur $\realIntervalOO{0}{+\infty}$ en posant $z(t)=y(\sqrt{t})$.
On a donc $y(x)=z(x^2)$  puis $y'(x)=2x z'(x^2 )$\dots\\
Résoudre également sur $\realIntervalOO{-\infty}{0}$.
\end{exercice}


\begin{exercice}
Déterminer les suites réelles $u$ vérifiant 
\[
\forall (n,k) \in\setNaturalNumbers^2, u_{n+k}=u_n u_k
\]
\end{exercice}
\begin{exercice}
Si X est une partie de N,on note $\mathcal{P}_1(X)$ et $\mathcal{P}_2(X)$ les propositions suivantes :
\[
\mathcal{P}_1(X): \forall x\in X,(\exists y\in X)\left/ x < y \right.
\]
\[
\mathcal{P}_2(X): (\exists x\in X)\left/\forall y\in\setNaturalNumbers, (x<y \Longrightarrow y \in X)\right.
\]
A-t-on $\mathcal{P}_1(X) \Longrightarrow\mathcal{P}_2(X)$? A-t-on $\mathcal{P}_2(X) \Longrightarrow\mathcal{P}_1(X)$?\\
Quel est le lien entre le fait que X est infini et les propriétés précédentes ?
\end{exercice}
\begin{exercice}
On note $\mathcal{P}(x,y)$ une propriété dépendant de deux réels $x$ et $y$.
On note A et B les propriétés suivantes :
\[
A=\forall x\in X (\exists y\in X)\left/\mathcal{P}(x,y)\right.
\]
\[
B=(\exists y\in X)\left/\mathcal{P}(x,y)\right.
\]
A implique-t-elle B ? B implique-t-elle A ? \\
On prend $\mathcal{P}(x,y)$ la propriété $x^2-xy+a=0$ avec a un réel. Pour quelles valeurs de a a-t-on A ? Idem avec B.
\end{exercice}
\begin{exercice}
Déterminer les $x\in\setRealNumbers$ tels que 
\[
\forall n\in\setNaturalNumbers,x^{n+2}\le x^{n+1}+x^n
\]
\end{exercice}
\begin{exercice}
Déterminer une condition nécessaire et suffisante sur $k\in\setIntegers$ pour qu'il existe $\theta\in\setRealNumbers$ tel que 
\[
e^{i\theta}+e^{ik\theta}=1
\]
\end{exercice}
\begin{exercice}
Donner une condition nécessaire et suffisante sur $a\in\setRealNumbers$ pour que la fonction $\function[f]{x}{x + a \sin(x)}$ soit une bijection de $\setRealNumbers$ dans $\setRealNumbers$.
\end{exercice}
\begin{exercice}
Déterminer une condition nécessaire et suffisante sur $z$ réel pour qu'il existe $(x,y)\in \setRealNumbers^2$  tels que 
\[
z=\dfrac{xy}{2x^2+y^2+1}
\]
\end{exercice}
\begin{exercice}
Déterminer une condition nécessaire et suffisante sur $\alpha\in\setComplexNumbers$ pour que 
\[
\forall(x,y)\in \setRealNumbers^2,(x+\alpha y=0 \Longleftrightarrow x=y=0)
\]
\end{exercice}

\chapter{Racines exactes des polynômes de degré  2, 3 et 4}
\subsection{Expression générale des racines d'un polynôme de degré 2}
Soient $(a,b,c)\in\setRealNumbers^3$. Soit $P = aX^2+bX+c$ un polynôme du second degré. Alors, cherchons les racines de $P$. 

Pour cela on doit résoudre l'équation 
\[
ax^2+bx+c = 0 \Longleftrightarrow x^2+\dfrac{b}{a}x+\dfrac{c}{a} = 0
\]
Or 
\[
\left(x+\dfrac{b}{2a}\right)^2=x^2+\dfrac{b}{a}x+\dfrac{b^2}{a^2}
\]
Donc 
\[
\left(x+\dfrac{b}{2a}\right)^2+\dfrac{c}{a}-\dfrac{b^2}{4a^2}=0
\]
Donc 
\[
\left(x+\dfrac{b}{2a}\right)^2=\dfrac{b^2-4ac}{4a^2}
\]
D'où 
\[
\left(x+\dfrac{b}{2a}\right) = \pm \dfrac{\sqrt{b^2-4ac}}{2a}
\]
Finalement, 
\[
x= -\dfrac{b}{2a}\pm \dfrac{\sqrt{b^2-4ac}}{2a}= \dfrac{-b\pm\sqrt{b^2-4ac}}{2a}
\]

\subsection{Expression générale des racines d'un polynôme de degré 3}
Soient $(a,b,c,d)\in\setRealNumbers^4$. Soit $P = aX^3+bX^2+cX+d$ un polynôme du troisième degré. Alors, cherchons les racines de $P$. \\ 
Pour cela on doit résoudre l'équation 
\[
ax^3+bx^2+cx+d=0
\]
C'est équivalent à 
\[
x^3+\dfrac{b}{a}x^2+\dfrac{c}{a}x+\dfrac{d}{a}=0
\]
On va éliminer le terme en $x^2$. Pour cela, on va utiliser le développement de $\left(x+\dfrac{b}{3a}\right)^3$.
Posons $\beta = x +  \dfrac{b}{3a}$ ,alors l'équation se transforme en 
\[
\left(\beta- \dfrac{b}{3a}\right)^3+\dfrac{b}{a}\left(\beta- \dfrac{b}{3a}\right)^2+\dfrac{c}{a}\left(\beta- \dfrac{b}{3a}\right)+\dfrac{d}{a}=0
\]
Donc 
\[
\beta^3+\underbrace{\left(\dfrac{b}{a}-3\dfrac{b}{3a}\right)}_{=0}\beta^2
+\underbrace{\left(\dfrac{c}{a}+(3-2)\dfrac{b}{a}\dfrac{b}{3a}\right)}_{p}\beta
+\underbrace{\left(\dfrac{d}{a}+\dfrac{b}{27a}\left(\dfrac{2b^2}{a^2}-\dfrac{9c}{a}\right)\right)}_{q} = 0
\]
On s'est donc ramené a l'équation $\beta^3+p\beta+q=0$\\
Prenons maintenant $u,v\in\setRealNumbers\left/ uv=-\dfrac{p}{q}\text{ et }\beta = u + v\right.$ (on va montrer par la suite que $u$ et $v$ sont bien définis)\\
Alors 
\[
u^3 v^3 = -\dfrac{p^3}{27}
\]
D'autre part, 
\[
\underbrace{(u+v)^3+p(u+v)}_{=-q}=u^3+v^3+3(u+v)uv+p(u+v)=u^3+v^3+\underbrace{3(uv+\dfrac{p}{3})(u+v)}_{=0}
\]
Donc $u^3+v^3 = -q$\\
On remarque astucieusement qu'on connait la somme et le produit de $u^3$ et $v^3$ donc ils vérifient l'équation polynômiale de degré 2 
\[
X^2 - \text{somme}X + \text{produit}=0 
\]
i.e. 
\[
X^2 +q X - \dfrac{p^3}{27} =0
\]
On a alors $\Delta = q^2+\dfrac{4p^3}{27}$. On ne connait pas le signe de $\Delta$ et il est parfois négatif. Ici on ne va traiter que le cas ou $\Delta$ est positifs, l'autre cas sera à mettre en relation avec le chapitre sur les complexes.
On connait les solutions de cette équation, on en déduit que \\
\[
u^3,v^3 = \dfrac{-q \pm \sqrt{q^2+\dfrac{4p^3}{27}}}{2}
\]
Donc 
\[
\beta = u + v = \sqrt[3]{\dfrac{-q - \sqrt{q^2+\dfrac{4p}{3}}}{2}} + \sqrt[3]{\dfrac{-q + \sqrt{q^2+\dfrac{4p^3}{27}}}{2}}
\]
Donc finalement 
\[
x = \sqrt[3]{\dfrac{-q - \sqrt{q^2+\dfrac{4p^3}{27}}}{2}} + \sqrt[3]{\dfrac{-q + \sqrt{q^2+\dfrac{4p^3}{27}}}{2}} - \dfrac{b}{3a}
\]

\subsection{Expression générale des racines d'un polynôme de degré 4}
Soient $(a,b,c,d,e)\in\setRealNumbers^5$\\
Soit $P = aX^4+bX^3+cX^2+dX+e$ un polynôme du quatrième degré. Alors, cherchons les racines de $P$.\\ 
Pour cela on doit résoudre l'équation 
\[ax^4+bx^3+cx^2+dx+e=0\]
Pour cela, posons $x = \beta - \dfrac{b}{4a}$. En remplaçant $x$ par son expression en fonction de $u$ et en développant on obtient l'équation
\[
u^4+\underbrace{\left(\dfrac{c}{a}-\dfrac{3b^2}{8a^2}\right)}_{p}u^2 + \underbrace{\left(\dfrac{d}{a}-\dfrac{bc}{2a^2}+\dfrac{b^3}{8a^3}\right)}_{q}u + \underbrace{\left(\dfrac{e}{a}-\dfrac{bd}{4a^2}+\dfrac{c b^2}{16a^3}-3\left(\dfrac{b}{4a}\right)^4\right)}_{r}= 0
\]
Donc 
\[
u^4+u^2+qu+r=0
\]
Cette forme d'équation fait fortement penser aux équations bicarrées, mais le terme en $u$ est gênant. On va essayer d'exprimer le polynôme sous une forme plus simple cependant.\\
Cherchons $\alpha, \beta, \gamma \in \setRealNumbers \left/(u^2+\alpha)^2+\beta (u+\gamma)^2=u^4+pu^2+qu+r\right.$\\
Pour cela, on va chercher des conditions en identifiant les monômes de degrés égaux. Ainsi, on obtient, 
\[
\left\{\begin{array}{l}
1=1\\
2\alpha+\beta = p\\
2 \beta \gamma = q\\
\alpha^2+\beta \gamma^2 = r\\
\end{array}\right.
\] 
La dernière équation nous donne après substitution des variables,
\[
\alpha^2+\beta \gamma^2=\alpha^2 + \dfrac{q^2}{4\beta}=\alpha^2 + \dfrac{q^2}{4(p-2\alpha)}=r
\]
Ainsi 
\[
-2\alpha^3+p\alpha^2+2r\alpha+\left(\dfrac{q^2}{4}-rp\right)=0
\]
Il s'agit d'une équation du troisième degré donc il existe toujours $\alpha$ qui vérifie cette relation. Cela nous donne automatiquement $\beta$ et $\gamma$. Il ne reste plus qu'à résoudre l'équation 
\[
(u^2+\alpha)=-\gamma(u+\beta)^2
\]
On obtient alors 2 équation du second degré, à savoir,
\[
u^2 + \sqrt{-\gamma}u + \left(\alpha + \sqrt{-\gamma}\right)
\]
Et
\[
u^2 - \sqrt{-\gamma}u + \left(\alpha - \sqrt{-\gamma}\right)
\]
Cela nous donne les solutions générales
\begin{proof}
\begin{equation}
u^2 - \sqrt{-\gamma}u + \left(\alpha - \sqrt{-\gamma}\right)
\end{equation}
\begin{equation}
u^2 - \sqrt{-\gamma}u + \left(\alpha - \sqrt{-\gamma}\right)
\end{equation}
\end{proof}
%\import{./}{importation.tex}

\chapter{test}

\begin{equation}
u^2 - \sqrt{-\gamma}u + \left(\alpha - \sqrt{-\gamma}\right)
\end{equation}
\begin{proof}
\begin{equation}
u^2 - \sqrt{-\gamma}u + \left(\alpha - \sqrt{-\gamma}\right)
\end{equation}
\end{proof}
\end{document}
